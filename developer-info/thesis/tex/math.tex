
\begin{chapter}{Mathematical Technique and Theory\label{chap:math}}
  
  The mathematical problem resulting from the physical system
  described in Chapter \ref{chap:physical} is, at first glance, a very
  simple one.  However, the calculation of matrix exponentials, such
  as the solution to the activation equation
  \begin{equation}
    \vec{N}(t) = e^{\mat{A} t} \vec{N}_o,\tag{\ref{eqn:intro.basic_soln}}
  \end{equation}
  are known to be difficult in general.  An extensive study of
  nineteen different methods found them all to be ``dubious'', saying
  that ``some of the methods are preferable to others, but that none
  are completely satisfactory.''\cite{DUBIOUS}
  
  If the original activation tree (without removing cross-links and
  straightening loops -- Figure \ref{fig:physical.basic_tree}) is
  converted directly to its mathematical equivalent, the result is a
  compact but potentially stiff system of linear first order ordinary
  differential equations [ODE's],
  \begin{align}
    \begin{split}
      \dot{\vec{N}}(t) &= \mat{A}\vec{N}(t)\\
        &= \left [
          \begin{array}{cccccc}
            -d_1 & P_{2\rightarrow 1} & P_{3\rightarrow 1} & \cdots & \cdots & P_{l\rightarrow 1}\\
            P_{1\rightarrow 2} & -d_2 & P_{3\rightarrow 2} & \cdots & \cdots & P_{l\rightarrow 2}\\
            P_{1\rightarrow 3} & P_{2\rightarrow 3} & -d_3 & \cdots & \cdots & P_{l\rightarrow 3}\\
            \vdots & \vdots & \vdots & \ddots &  & \vdots\\
            \vdots & \vdots & \vdots & & -d_{l-1} & P_{l\rightarrow l-1}\\
            P_{1\rightarrow l} & P_{2\rightarrow l} & P_{3\rightarrow l} & \cdots & P_{l-1\rightarrow l} & -d_l
          \end{array} \right ] \cdot \left [
          \begin{array}{c}
            N_1 \\ N_2 \\ N_3 \\ \vdots \\ \vdots \\ N_l
          \end{array} \right ]_,
    \end{split}\label{eq:math.intro.explicit}\\
    \intertext{where:}
    \vec{N} &\equiv \text{number densities, $N_i$, of all isotopes}\notag \\
    d_i &\equiv \text{destruction rate of isotope $i$}\notag \\
    P_{i\rightarrow j} &\equiv \text{production rate of isotope $j$ from isotope $i$.}\notag
  \end{align}
  
  After loop straightening and cross-link removal is performed (see
  figure \ref{fig:physical.straight_tree}), the result is a somewhat
  larger, simpler set of ODE's:
  \begin{equation}
    \begin{split}
      \dot{\vec{N}}(t) &= \mat{B}\vec{N}(t)\\
        &= \left [
          \begin{array}{cccccc}
            -d_1 & 0 & 0 & \cdots & \cdots & 0 \\
            P_{1\rightarrow 2} & -d_2 & 0 & \cdots & \cdots & 0\\
            P_{1\rightarrow 3} & P_{2\rightarrow 3} & d_3 & \cdots & \cdots & 0 \\
            \vdots & \vdots & \vdots & \ddots &  & \vdots\\
            \vdots & \vdots & \vdots & & -d_{m-1} & 0\\
            P_{1\rightarrow m} & P_{2\rightarrow m} & P_{3\rightarrow m} & \cdots & P_{m-1\rightarrow m} & -d_m
          \end{array} \right ] \cdot \left [
          \begin{array}{c}
            N_1 \\ N_2 \\ N_3 \\ \vdots \\ \vdots \\ N_m
          \end{array} \right ]_.\label{eq:math.intro.tree}
    \end{split}
  \end{equation}
  This lower triangular matrix is quite sparse with a maximum of two
  entries in each row since each \pc-node has only one production path
  and one total destruction rate.
  
  Finally, if this physical model is further broken into the
  previously described linear chains (figure
  \ref{fig:physical.chains}), many small sets of ODE's are created
  with special simplifying characteristics,
  \begin{equation}
    \begin{split}
      \dot{\vec{N}}(t) &= \mat{C}\vec{N}(t)\\
        &= \left [
          \begin{array}{cccccc}
            -d_1 & 0 & 0 & \cdots & \cdots & 0 \\
            P_{1\rightarrow 2} & -d_2 & 0 & \cdots & \cdots & 0\\
            0 & P_{2\rightarrow 3} & -d_3 & \cdots & \cdots & 0 \\
            0 & 0 & P_{3\rightarrow 4} & -d_4 &  & \vdots\\
            \vdots & \vdots & \vdots & & \ddots & 0\\
            0 & 0 & 0 & \cdots & P_{k-1\rightarrow k} & -d_k
          \end{array} \right ] \cdot \left [
          \begin{array}{c}
            N_1 \\ N_2 \\ N_3 \\ \vdots \\ \vdots \\ N_k
          \end{array} \right ]_.\label{eq:math.intro.chains}
    \end{split}
  \end{equation}
  The bidiagonal nature of these matrices is an important factor in
  subsequent derivations and calculations.

  In all cases, the generic solution takes the form
  \begin{equation}
    \vec{N}(t) = \mat{T}\vec{N}_o(t),\label{eq:math.intro.main}
  \end{equation}
  where \mat{T} is the exponential of the matrices \mat{A}, \mat{B},
  or \mat{C}, depending on which method is used (\textsl{e.g.}
  $\mat{T} = e^{\mat{A} t}$).
  
  It is interesting to compare the sizes of these three matrices as it
  gives some initial insight into the efficiency of the solution.  To
  compare the size of the original tree, $l$, with that of the
  straightened tree, $m$, is not easy.  Since the physical conversion
  is to convert cross-links and loops into \pc-nodes, it is clear that
  $m > l$; however, the severity of this inequality is difficult to
  determine.  Nevertheless, an approximate comparison between $k$ and
  $m$ can be made.  Since \mat{B} represents a true tree structure, it
  can be analyzed by assuming that it is a balanced $n$-ary tree, that
  is, assuming that every \pc-node in the tree has exactly $n$
  branches.  Since $k$ is the depth of such a tree, there will be
  $n^{k-1}$ chains representing the $m = \frac{n^k -1}{n-1} \approx
  O(n^{k-1})$ nodes of the tree.  The computational complexity of the
  mathematical operations performed on these matrices are generally at
  least of the order of the square of the matrix dimension, and often
  the cube.  Thus, for matrix \mat{B}, the mathematical costs will be
  at least $O(n^{2k-2})$ and possibly $O(n^{3k-3})$ or higher.  On the
  other hand, for the $n^{k-1}$ matrices \mat{C}, the mathematical
  costs will be $O(k^2n^{k-1})$ or $O(k^3n^{k-1})$.  This very rough
  analysis shows that for mathematical operations of order $x$, as
  long as $n^{1-\frac{1}{x}} > k^{\frac{1}{k-1}}$, the linear chain
  method will be more efficient.  While it may be difficult to
  visualize this relationship, it can be used to define some limits.
  
  If the mathematical operations have second order computational
  complexity, the linear chain method is always more efficient if $n
  \geq 4$ (there are at least 4 branches per \pc-node).  If this
  complexity increases to third order, the physical model only
  requires $n \geq 3$ for linear chains to be more efficient.  From
  another perspective, if the number of branches per \pc-node is fixed
  at $n = 2$, and chain depth, $k = 3$, operations with $5^{th}$ order
  complexity are required for the linear chains to be more efficient.
  This requirement decreases very quickly for longer chains, with only
  third order complexity required at $k = 4$.  Finally, as with most
  real problems, $n \geq 4$ and the linear chain method is more
  efficient for all orders of complexity in the mathematical
  operations.
  
  It is important to note that not only is this analysis not rigorous,
  but the order of the mathematical solution is itself dependent on
  the method which is used.  Thus, the analysis gives a first glance
  into the comparison of efficiency, but is hardly complete.
  
  Historically, a wide range of matrix methods, non-matrix time-step
  based solvers and matrix/time-step hybrid solvers have been used.
  
  For the non-matrix methods, the time history of the problem is
  divided into time steps and the equations are solved using some
  variation of a standard differencing technique, including
  Runge-Kutta, Euler, or GEAR methods.  By choosing small enough time
  steps, the numerical error of such methods can be reduced to make
  them sufficiently accurate.  The consequence of using small enough
  time steps, however, is increased computational time.
  
  Hybrid methods are variations on ``scaling and squaring'' where some
  small time step is chosen, either based on the time constants of the
  problem, or by taking sufficient square roots of the total time of
  interest.  The scaled matrix exponential is solved with a rapidly
  converging series solution and then raised to some power, or
  successively squared, to reach the full operation time.  These
  methods are both computationally inefficient, and in the general
  case, can be prone to round-off error.
  
  It is also possible to use matrix decomposition methods to solve
  this formulation.  The matrix exponential can be decomposed as,
  $\mat{A} = \mat{S}\mat{D}\mat{S}^{-1},$ giving the solution
  $e^{\mat{A} t} = \mat{S}e^{\mat{D} t}\mat{S}^{-1},$.  The goal of
  this class of methods is to find a decomposition such that the
  exponential $e^{\mat{D} t}$ is easily calculated. Some matrix
  decomposition methods which have been explored for this application
  include eigenvector decomposition\cite{eigenvectors}, generalized
  eigenvector decomposition\cite{GERAPH} or Schur
  decomposition\cite{RACCP}.  For some special forms of the transfer
  matrix, \mat{A}, these decomposition methods are accurate.  However,
  when formulated for the general case, the decompositions can be
  prone to round-off error.
  
  Within the class of matrix methods, however, implementations
  designed to take advantage of the special nature of the linear chain
  matrix \mat{C}, can be promising.  The individual ODE's are solved
  with the assistance of Laplace transforms, filling the resultant
  matrix exponential one element at a time.  These methods are
  versions of matrix decomposition methods which have been
  reformulated to minimize round-off errors and handle defective
  matrices, so much so that they are hardly recognizable as matrix
  decomposition methods.  Previous applications have used a
  formulation of the analytic Bateman\cite{DKR,DKRICF,DKRP} solution
  for linear chains, but this is only applicable when no loops occur.
  A new method, developed in this work, is able to replace the Bateman
  solution, in the event that a loop occurs in the linear chain.
  
  One primary advantage of matrix methods is their ability to quickly
  solve problems with irradiation histories based on repetition,
  whether it be simple hierarchies of pulsing, or complex hierarchies
  of schedules and sub-schedules.  A single matrix must be solved for
  each relevant flux level and/or characteristic time, and these
  matrices are multiplied appropriately to build the response matrix
  for the entire history.  Time-step methods are unable to do this as
  they never generate a transfer matrix, but work directly with the
  isotopic number density vector.  Since it is impossible to
  determine a unique transfer matrix given the initial and final
  vector, time-step methods are forced to simply step through the
  entire history, applying the whole array of calculations at each
  time step and modeling each pulse individually.
  
  By transferring this system to the Laplace domain and considering
  each equation individually, it is possible to write the solution in
  a more compact form (N.B. $P_{i}$ implies $P_{i-1 \rightarrow i}$):
  \begin{align}
    \tilde{N}_i &= \frac{N_{i_0}}{s+d_i} + P_i
    \frac{\tilde{N}_{i-1}}{s+d_i}\\
    \begin{split}
      &= \frac{N_{i_0}}{s+d_i} + 
      \frac{N_{i-1_0}}{s+d_{i-1}}\frac{P_i}{s+d_i} +
      \frac{N_{i-2_0}}{s+d_{i-2}}\frac{P_{i-1}P_i}{(s+d_{i-1})(s+d_i)} + 
      \cdots\\
      &\quad+\frac{N_{2_o}}{s+d_2}\prod_{j=3}^i\frac{P_{j}}{s+d_{j}} +
      \frac{N_{1_o}}{s+d_1}\prod_{j=2}^i\frac{P_j}{s+d_{j}},\\
    \end{split}\\
    \intertext{which can be written as}
    \begin{split}
      \tilde{N}_i & = \sum_{j=1}^i \tilde{N}_{ij} \\
      & = \sum_{j=1}^i N_{j_o} \prod_{k=j+1}^i P_k \prod_{l=j}^i
      \frac{1}{s+d_l} \\
      & = \sum_{j=1}^i N_{j_o} \tilde{F}_{ij}(s)
      \prod_{k=j+1}^i P_k.\label{eq:math.full.prob}\\
    \end{split}\\
    \intertext{In this representation, the matrix \mat{T} is filled by setting}
    \begin{split}
      T_{ij} &= N_{ij}/N_{j_o}\\ &=
      \mathcal{L}^{-1}\left[\tilde{F}_{ij}(s)\right] \prod_{k=j+1}^{i}
      P_k,\\
    \end{split}\label{eq:math.intro.filler}\\
    \intertext{and it becomes an exercise of solving for the inverse
      transform of the term}
    \tilde{F}_{ij}(s) &= \prod_{l=j}^i \frac{1}{s+d_l}\label{eq:math.root.prob}.
  \end{align}
  
  Normally, two such matrices, \mat{T} and \mat{D}, are required to
  represent the pulse and dwell times, respectively.  In the first
  case, all the destruction and production rates include the terms for
  neutron transmutation which are dependent on the flux spectrum and
  therefore it is only during this period in which loops can occur in
  the isotope tree.  In the dwell period, the destruction and
  production rates are only those of decay, and therefore, many of the
  values will be zero.

  \begin{section}{The Analytical Bateman Solution}\label{sec:math.bateman}
    
    If all the destruction rates, $d_i$, are distinct, Equation
    \ref{eq:math.root.prob} can be easily inverted,
    \begin{equation}
      f_{ij}(t) = \sum_{l=j}^{i} e^{-d_l t} 
      \prod_{\substack{
          m=j \\
          m\neq l
          }}^i
      \frac{1}{d_m - d_l}\label{eqn:math.bateman.inversion}.
    \end{equation}
    This leads to a compact representation of the solution to the
    Bateman equations:
    \begin{equation}
      N_i(t) = N_{i_o}e^{-d_i t} + \sum_{j=1}^{i-1}N_{j_o}\left [
        \sum_{k=j}^{i-1}\frac{P_{k+1}(e^{-d_k t} - e^{-d_i t})}{d_i -
          d_k}\prod_{\substack{l=j\\l\neq k}}^{i-1}\frac{P_{l+1}}{d_l-d_k}\right] \, .
    \end{equation}
    Finally, we can write the transfer matrix elements as:
    \begin{equation}
      \begin{split}
        T_{ii} &= e^{-d_i t}\\
        T_{\substack{ij\\i\neq j}} &= \sum_{k=j}^{i-1}\frac{P_{k+1}(e^{-d_k t} - e^{-d_i t})}{d_i -
          d_k}\prod_{\substack{l=j\\l\neq k}}^{i-1}\frac{P_{l+1}}{d_l-d_k}.\label{eqn:math.bateman.final}\\
      \end{split}
    \end{equation}
  
    It it can be shown that the mathematical operations performed in
    this method are essentially the same operations that would be
    carried out if an eigenvector decomposition was used to solve the
    matrix.

  \end{section}

  \begin{section}{Laplace Inversion Method}\label{sec:math.inversion}
  
    When the destruction rates (eigenvalues) are not distinct, other
    methods of solving equation \ref{eq:math.root.prob} are required.
    In general, however, because of the bidiagonal nature of the
    matrix representation (that is, because of the linear chain
    physical representation), this is a simple problem which, for a
    small system, can easily be solved on paper by hand.  In
    particular, this Laplace space representation can be directly
    inverted to the time representation.
  
    For repeated poles, one uses the residue theorem to determine the
    coefficients for each term in a partial fractions expansion.  Each
    of those terms would result in an exponential, perhaps multiplied
    by a polynomial in time, $t$, when converted back to the time
    domain.  Those residues are calculated using one of two simple
    rules.
  
    If the pole is not repeated, then the residue, $R_k$, for the pole,
    $-d_k$, is calculated as
    \begin{equation}
      R_k = \lim_{s\rightarrow d_k} (s+d_k) \tilde{F}_{ij}(s)
    \end{equation}
    which becomes $R_k e^{-d_kt}$ in the time domain.  If all poles have a
    singular multiplicity, the solution reduces exactly to the Bateman
    solution, and can be represented in many ways.  This is obviously the
    solution with no loops.
    
    If the pole is repeated $m$ times, under a partial fraction
    expansion, this becomes $m$ terms in that expansion:
    \begin{equation}
      \frac{R_{km}}{(s+d_k)^m} + \frac{R_{km-1}}{(s+d_k)^{m-1}} +\cdots + \frac{R_{k1}}{(s+d_k)}
    \end{equation}
    which becomes 
    \begin{equation}
      e^{-d_kt} \left ( R_{km} \frac{t^{m-1}}{(m-1)!} + R_{k,m-1}
        \frac{t^{m-2}}{(m-2)!} + \cdots + R_{k2} \frac{t}{1!} + R_{k1} \right )
    \end{equation}
    in the time domain.  In this case, the residues are found using:
    \begin{equation}
      R_{kn} = \frac{1}{(m-n)!} \lim_{s\rightarrow d_k}
      \frac{d^{m-n}}{ds^{m-n}} \left [ (s+d_k)^m \tilde{F}_{ij}(s) \right].
    \end{equation}
    
    This latter rule requires the ability to evaluate derivatives of a
    generic function:
    \begin{equation}
      \tilde{G}_{ij}^k(s) = (s+d_k)^m \tilde{F}_{i,j}(s)
    \end{equation}
    at values of $s = -d_k$ for all $i$.  By examining the successive
    derivatives of $\tilde{G}(s)$ it can be shown (see Appendix
    \ref{app:math.deriveG}) that these derivatives can be recursively
    defined as:
    \begin{equation}
      \left[\tilde{G}_{ij}^k(s)\right]^{(n)} = \sum_{j=1}^n(-1)^j
      \frac{(n-1)!}{(n-j)!}\left[\tilde{G}_{ij}^k(s)\right]^{(n-j)}
      \sum_{i=1}^I \frac{1}{(s+d_i)^j}
    \end{equation}
    and this, in turn, can be converted to a computational numerical
    algorithm, allowing the entire problem to be solved.
  
    We will call this method the Laplace Inversion Method.

  \end{section}  
  
  \begin{section}{Laplace Expansion Method}\label{sec:math.expansion}
    
    Both of the above analytical solutions can be prone to round-off
    error when the absolute difference between two poles is small,
    either because the poles are nearly identical or because both
    poles are very small.  Each term in the sum will be very large due
    to division by small numbers, and successive terms will differ
    only in the least significant digits.  For these cases, an
    expansion in $1/s$ may be preferred.
    
    It is apparent from equation \ref{eqn:math.bateman.final},
    however, that such divisions can be eliminated by writing the
    solution as a difference of exponentials, using the series
    expansions for the exponentials, factoring out the offending term
    from the numerator and canceling.  While this seems like a
    monumental task to perform on an arbitrary problem, thanks to the
    bidiagonal nature of the system, it is again quite simple to
    implement.  If we start again with our function:
    \begin{equation}
      \tilde{F}_{ij}(s) = \prod_{l=j}^{i}\frac{1}{s+d_l}
    \end{equation}
    and making no assumptions about the multiplicity of the poles, we
    expand this as a series in $1/s$, the result is:
    \begin{align}
      \begin{split}
        \tilde{F}_{ij}(s) &=\frac{1}{s^{i-j+1}} \prod_{l=j}^i\frac{1}{1+\frac{d_l}{s}}\\
        &=\frac{1}{s^{i-j+1}} \prod_{l=j}^i 
        \left(1 - \frac{d_l}{s} + \frac{d_l^2}{s^2} - \frac{d_l^3}{s^3} + \cdots \right )\\
        &= \frac{1}{s^{i-j+1}} \left [ 1 - \frac{\sum_{l=j}^i d_l}{s} +
          \frac{\sum_{l=j}^{i}d_l\sum_{k=l}^i d_k}{s^2} - 
          \frac{\sum_{l=j}^{i}d_l \sum_{k=l}^i d_k \sum_{m=k}^i d_m}{s^3} + \cdots
        \right ].
      \end{split}\\
      \intertext{If $n=i-j$, in the time domain, this becomes:}
      \begin{split}
        f_{ij}(t) &= t^{n} \left[ \frac{1}{n!} - \frac{t}{(n+1)!}\sum_{l=j}^i d_l 
          +\frac{t^2}{(n+2)!}\sum_{l=j}^{i}d_l \sum_{k=l}^i d_k\right.\\  
        &\phantom{= t^{n-1} \left[\right.}
        \left.\quad\quad-\frac{t^3}{(n+3)!}\sum_{l=j}^{i}d_l \sum_{k=l}^i d_k \sum_{m=k}^i d_m 
          +\cdots \right ]\label{eqn:math.expansion.final}\\
      \end{split}\\
      \intertext{and thus:}
      \begin{split}
        T_{ii} &= e^{-d_i t}\\
        T_{\substack{ij\\i\neq j}} &= f_{ij}(t)\prod_{k=j}^{i-1}P_k \, .\\
      \end{split}
    \end{align}
    
    It is clear that this solution will only be computationally viable
    when the product, $\max \{d_i\} \cdot t$ is small.  For arbitrary
    problems, this is only guaranteed when there are short operating
    times, but can easily be tested for a particular problem.  Other
    representations can be formed, each providing different insight
    into the method (see Appendix \ref{app:math.1_s}).
    
    We will call this method the Laplace Expansion Method.

  \end{section}
  
  \begin{section}{Adaptive Mathematical Methods\label{sec:math.adaptive}}
  
    Upon examining available methods to solve such lower bidiagonal
    systems, there are certain easily determined matrix
    characteristics which can be used to adaptively choose a method
    for each matrix element which will optimize the speed and accuracy
    of the solution.
  
    The primary criterion for choosing amongst the solution methods is
    the presence of degenerate eigenvalues.  The eigenvalues of these
    matrices are simply the diagonal elements, which in turn are the
    destruction rates of each isotope in the linear chain being
    modeled.  True degeneracies in these values will occur only when
    the chain being modeled by this matrix has straightened loops and
    a simple test can determine whether this is the case.  Each matrix
    element, $T_{i,j}$ represents the transfer within a segment of the
    chain between nodes $j$ and $i$.  Even if loops exist elsewhere in
    the chain, if no loops occur in this chain segment, the analytical
    Bateman solution can be selected.
  
    If a loop does exist within this sub-chain, the laplace expansion
    technique is immediately implemented.  When, during the course of
    the expansion solution, it does not converge quickly enough, the
    laplace inversion technique is invoked.
  
    For the dwell periods between pulses, the Bateman solution can
    always be used because the absence of transmutation reactions
    makes loops physically impossible.
  
    Because the mathematical method is chosen independently and
    adaptively for each matrix element, the efficiency and accuracy of
    the entire solution is optimized.  If a small loop occurs in one
    small portion of an activation tree, it is not necessary to
    perform the relatively computationally intensive Laplace Expansion
    and/or Inversion solution on the entire problem.  On the other
    hand, the solution is not limited by the non-loop Bateman solution
    when loops do exist.

  \end{section}
  
  \begin{section}{\ALARA\  Implementation Summary}\label{sec:math.implement}
  
    As with the physical modeling, the implementation of the
    mathematical solution requires some special implementation to
    enhance its efficiency.
  
    \enlargethispage*{\baselineskip}
    The first such enhancement is to store the solution matrices from
    one chain to another.  The value of this implementation can be
    seen when considering the solution of many subsequent chains of
    large rank, $n$.  Without saving the previously generated
    matrices, all the $n(n-1)/2$ elements of the lower-triangular
    transfer matrices would have to be completely recalculated
    ($n^2/2$ calculations) for each of these subsequent chains, even
    if they only differ in the last rank.  When the transfer matrix
    solution for each interval is stored after its use in the solution
    of one chain, only the last row ($n$ calculations) is necessary.
    This savings carries to all situations throughout the activation
    tree.  Since the chains are formed by a depth-first search, it is
    likely that for each chain, a significant portion of the data has
    already been determined for a previous calculation.  Since the
    physical model allows a complex hierarchy of irradiation schedules
    and sub-schedules, the storage for these transfer matrices must
    mimic this hierarchy.  Furthermore, as the reaction rates are a
    function of the neutron flux, one of these storage hierarchies is
    required for each interval.
    
    \enlargethispage*{\baselineskip}
    This is also true for the decay matrices between pulses and after
    shutdown, however in this case, the matrices are not only saved
    from one chain to the next, but across all the intervals being
    solved for the same chain.  Since the decay matrices are
    independent of flux, they need only be calculated once for each
    chain and then used in all the intervals for that chain.  This has
    potentially large savings since there may be many intervals
    sharing the decay matrices, each recalculation of which costing
    $n$ calculations even with the already implemented savings.
    Furthermore, since the decay matrices are likely to be much
    sparser than the pulse transfer matrices, a special implementation
    of the Bateman solution routine to intelligently fill these
    matrices has been implemented.  If any single production rate is
    equal to zero in the chain segment relevant to a particular matrix
    element, that matrix element will be zero, and there is no point
    in performing the full Bateman solution.
  
    These mathematical methods are implemented during a depth-first
    search of the irradiation history hierarchy.  A storage block is
    allocated for each item in each schedule in the hierarchy,
    consisting of the item's operation block transfer matrix and the
    total transfer matrix, including the effect of the operation
    block, the pulsing hierarchy and the dwell time between this block
    and the next.  For a simple pulse the operation block matrix is
    the transfer matrix for the pulse itself while the total transfer
    matrix includes the effect of the pulsing hierarchy and the
    inter-item dwell period.  For a complex sub-schedule, the
    operation block is calculated each time the depth-first search
    retracts back up the schedule hierarchy as the product of total
    transfer matrices of that sub-schedule's items.  The total
    transfer matrix again combines the effects of the operation block,
    the pulsing hierarchy and the inter-item dwell period.
  
    The solution of a particular pulsing hierarchy deserves a little
    more attention.  Given the transfer matrix of the operation block,
    say $\mat{T}_0$, and a single dwell matrix for the dwell time of
    the first level of pulsing, $\mat{D}_1$.  The product of these,
    $\mat{D}_1\mat{T}_0$, is then raised to a power representing one
    less than the number of pulses in that level, $n_1$.  Multiplying
    by the operation block matrix one more time becomes the transfer
    matrix for the next level of pulsing, $\mat{T}_1 =
    \mat{T}_0(\mat{D}_1\mat{T}_0)^{n_1}$.  This is repeated with the
    single dwell matrix for the dwell time of the second level,
    $\mat{D}_2$, and so on, using the general formula\cite{spangler}
    \begin{equation}
      \mat{T}_i = \mat{T}_{i-1}(\mat{D}_{i}\mat{T}_{i-1})^{n_i}.
      \label{eqn:pulsing.recurse}
    \end{equation}
    It is important to use an efficient algorithm\cite{UofT_csc280}
    for this matrix exponentiation process since it will be performed
    so often during the operation of the code.  For $N$ levels of
    pulsing, the matrix, $\mat{T}_{N}$, is the transfer matrix for the
    entire pulsing hierarchy up to the dwell time after that item.
    
    Using figure \ref{fig:physical.sample_hist} as an example,
    $\mat{T}_0$ represents the activation solution for one 10 minute
    pulse and $\mat{D}_1$ represents the solution for one 20 minute
    dwell period.  Therefore, $\mat{T}_1 =
    \mat{T}_0(\mat{D}_1\mat{T}_0)^{15}$ represents the solution for
    one work-day's worth of pulsing.  If $\mat{D}_2$ represents the
    solution for one dwell period of 16 hours and 20 minutes, then
    $\mat{T}_2 = \mat{T}_1(\mat{D}_2\mat{T}_1)^{4}$ is the solution
    for a full 5 day week of experiments.  This continues until the
    last level where $\mat{T}_4 = \mat{T}_3(\mat{D}_4\mat{T}_3)^{19}$
    represents the solution for 20 years of operation.
    
  \end{section}
  
\end{chapter}

