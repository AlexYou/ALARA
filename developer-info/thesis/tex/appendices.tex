\appendix

\begin{chapter}{Derivation of Recursive Derivative Definition\label{app:math.deriveG}}
    \begin{align}
    G(s)      &= \prod_{i=1}^N\frac{1}{s+d_i}\\
    \begin{split}
      G'(s)     &= \sum_{j=1}^N \frac{-1}{s+d_j} \prod_{i=1}^N\frac{1}{s+d_i}\\
                &= -G(s) \sum_{j=1}^N \frac{1}{s+d_j}\label{eqn:Gp}
    \end{split}\\
    G''(s)    &=
         -G'(s)\sum_{j=1}^N \frac{1}{s+d_j} + G(s)\sum_{j=1}^N(s+d_j)^{-2}\label{eqn:Gpp}\\
    \begin{split}
      G'''(s)  &= -G''(s)\sum_{j=1}^N \frac{1}{s+d_j} 
                + G'(s)\sum_{j=1}^N(s+d_j)^{-2}\\
               &\quad+G'(s)\sum_{j=1}^N(s+d_j)^{-2}  
                - 2G(s)\sum_{j=1}^N(s+d_j)^{-3}\\
               &=-G''(s)\sum_{j=1}^N \frac{1}{s+d_j}
                + 2G'(s)\sum_{j=1}^N(s+d_j)^{-2}\\
               &\quad-2G(s)\sum_{j=1}^N(s+d_j)^{-3}
    \end{split}\\
    \begin{split}
      G''''(s)  &=-G'''(s)\sum_{j=1}^N \frac{1}{s+d_j}+G''(s)\sum_{j=1}^N(s+d_j)^{-2}\\
                &\quad+2G''(s)\sum_{j=1}^N(s+d_j)^{-2}-4G'(s)\sum_{j=1}^N(s+d_j)^{-3}\\
                &\quad-2G'(s)\sum_{j=1}^N(s+d_j)^{-3}+6G(s)\sum_{j=1}^N(s+d_j)^{-4}\\
                &=-G'''(s)\sum_{j=1}^N\frac{1}{s+d_j}+3G''(s)\sum_{j=1}^N(s+d_j)^{-2}\\
                &\quad-6G'(s)\sum_{j=1}^N(s+d_j)^{-3}+6G(s)\sum_{j=1}^N(s+d_j)^{-4}
    \end{split}
  \end{align}
  Thus, for n=4,
  \begin{equation}
    \begin{split}
      G^{(n)}(s) &=-\frac{(n-1)!}{(n-1)!}G^{(n-1)}(s)\sum_{j=1}^N \frac{1}{s+d_j}\\ 
                 &\quad+\frac{(n-1)!}{(n-2)!} G^{(n-2)}(s)\sum_{j=1}^N(s+d_j)^{-2}\\ 
                 &\quad-\frac{(n-1)!}{(n-3)!} G^{(n-3)}(s)\sum_{j=1}^N(s+d_j)^{-3}\\
                 &\quad+\frac{(n-1)!}{(n-4)!} G^{(n-4)}(s)\sum_{j=1}^N(s+d_j)^{-4}\\
                 & = \sum_{i=1}{n} (-1)^i \frac{(n-1)!}{(n-i)!} G^{(n-i)}(s)\sum_{j=1}^N(s+d_j)^{-i}
    \end{split}
  \end{equation}

  \begin{section}{Induction Proof}

    \begin{align}
      G^{(n)}(s) &= \sum_{i=1}^n (-1)^i \frac{(n-1)!}{(n-i)!} G^{(n-i)}(s)
                    \sum_{j=1}^{N} (s+d_j)^{-i}\\
      \intertext{given}
      G^{(0)}(s) &= G(s) = \prod_{j=1}^N (s+d_j)^{-1}\\
      \intertext{First, we solve for n=1:}
      \begin{split}
        G'(s) & =  (-1) \frac{0!}{0!} G(s) \sum_{j=1}^N (s+d_j)^{-1}\\
             & =  -G(s)  \sum_{j=1}^N (s+d_j)^{-1}
      \end{split}
    \end{align}
    \noindent which matches Equation \ref{eqn:Gp}.

    \vspace{1cm}

    \noindent Now, we solve for n=2:
    \begin{equation}
      \begin{split}
        G''(s) & =  (-1)\frac{1!}{1!}G'(s)\sum_{j=1}^N (s+d_j)^{-1} +
                    \frac{1!}{0!}G(s)\sum_{j=1}^N (s+d_j)^{-2}\\
               & =  -G'(s)\sum_{j=1}^N (s+d_j)^{-1} + G(s)\sum_{j=1}^N (s+d_j)^{-2}
      \end{split}
    \end{equation}
    \noindent which matches Equation \ref{eqn:Gpp}.

    \vspace{1cm}

    \noindent Now, given $G^{(k)}(s)$, we take the derivative,
    $G^{(k+1)}(s)$, and see if it matches the correct form:
    \begin{align}
      G^{(k+1)}(s) &= \sum_{i=1}^k (-1)^i \frac{(k-1)!}{(k-i)!} \left [
                      G^{(k-i+1)} \sum_{j=1}^N (s+d_j)^{-i} - iG^{(k-i)}\sum_{j=1}^N
                      (s+d_j)^{-(i+1)} \right ]\\
      \intertext{letting $l = k+1$:}
      \begin{split}
        G^{(l)}(s) &=\quad \sum_{i=1}^{l-1} (-1)^i \frac{(l-2)!}{(l-i-1)!} G^{(l-i)} \sum_{j=1}^N (s+d_j)^{-i} \\
                   &\quad-\sum_{i=1}^{l-1} (-1)^i\frac{(l-2)!}{(l-i-1)!}iG^{(l-i-1)}\sum_{j=1}^N (s+d_j)^{-(i+1)}
      \end{split}\\
      \intertext{Now, letting $m=i+1$ in the second sum:}
      \begin{split}
        G^{(l)}(s) & = \quad\sum_{i=1}^{l-1} (-1)^i \frac{(l-2)!}{(l-i-1)!}G^{(l-i)} \sum_{j=1}^N (s+d_j)^{-i} \\
                   &\quad+ \sum_{m=2}^{l} (-1)^m\frac{(l-2)!}{(l-m)!}(m-1)G^{(l-m)}\sum_{j=1}^N (s+d_j)^{-m}
      \end{split}\\
      \intertext{and recombining the sums:}
      \begin{split}
        G^{(l)}(s) &=-\frac{(l-2)!}{(l-2)!}G^{(l-1)}(s)\sum_{j=1}^N (s+d_j)\\ 
                   &\quad+ \sum_{i=2}^{l-1} (-1)^i \left [
                         \frac{(l-2)!}{(l-i-1)!} + (i-1)\frac{(l-2)!}{(l-i)!} \right
                         ]G^{(l-i)}\sum_{j=1}^N (s+d_j)^{-i}\\
                   &\quad+ (-1)^l (l-2)!(l-1)G(s)\sum_{j=1}^N (s+d_j)^{-l}
      \end{split}\\
      \begin{split}
        &=-G^{(l-1)}(s)\sum_{j=1}^N (s+d_j)^{-1}\\
        &\quad+ \sum_{i=2}^{l-1} (-1)^i \left [
          (l-i)\frac{(l-2)!}{(l-i-1)!(l-i)} + (i-1)\frac{(l-2)!}{(l-i)!} \right ]
          G^{(l-i)}\sum_{j=1}^N (s+d_j)^{-i}\\
        &\quad+ (-1)^l (l-1)!G(s)\sum_{j=1}^N (s+d_j)^{-l}
      \end{split}\\
      \begin{split}
        &=-G^{(l-1)}(s)\sum_{j=1}^N (s+d_j)^{-1}\\
        &\quad+\sum_{i=2}^{l-1} (-1)^i
            \frac{(l-1)!}{(l-i)!}G^{(l-i)}\sum_{j=1}^N (s+d_j)^{-i}\\
        &\quad+ (-1)^l (l-1)!G(s)\sum_{j=1}^N (s+d_j)^{-l}
      \end{split}\\
      & =  \sum_{i=1}^{l} (-1)^i \frac{(l-1)!}{(l-i)!}G^{(l-i)}\sum_{j=1}^N
           (s+d_j)^{-i}
    \end{align}

    QED.

  \end{section} %% induction proof

\end{chapter} %% deriving the recursive derivatives

\begin{chapter}{Other Forms  of $1/s$ Expansion\label{app:math.1_s}}
  
  The $1/s$ expansion from section \ref{sec:math.expansion} can take
  on many slightly different forms providing different methods for
  determining the coefficients.  First, it is instructive to relate
  the expansion as shown in equation \ref{eqn:math.expansion.final} to
  a simple difference of exponentials.  Starting with the Bateman
  solution (equation \ref{eqn:math.bateman.inversion}) for a single
  matrix element,

  \begin{align}
    T_{31} &= \frac{P_2 (e^{-d_1 t} - e^{-d_3 t})}{d_3 - d_1}\frac{P_3}{d_2 - d_1} +
              \frac{P_3 (e^{-d_2 t} - e^{-d_3 t})}{d_3 - d_2}\frac{P_2}{d_1 - d_2}\tag{\ref{eqn:math.bateman.final}}\\
    \intertext{and using the standard expansion for the exponential, we get}
    \begin{split}
      &= P_2 P_3 \left [ \frac{1 - d_1 t + \frac{(d_1 t)^2}{2} - \frac{(d_1 t)^3}{6} 
              - 1 + d_3 t - \frac{(d_3 t)^2}{2} + \frac{(d_3 t)^3}{6} + \ldots }
              {(d_3 - d_1)(d_2 - d_1)} \right.\\
      &\phantom{= P_2 P_3 \left [ \right .}
              \left .\quad+\frac{1 - d_2 t + \frac{(d_2 t)^2}{2} - \frac{(d_2 t)^3}{6} 
              - 1 + d_3 t - \frac{(d_3 t)^2}{2} + \frac{(d_3 t)^3}{6} + \ldots }
              {(d_3 - d_2)(d_1 - d_2)} \right]\\
      &= P_2 P_3 \left [ \frac{(d_3 - d_1) [t - (d_3 + d_1)\frac{t^2}{2} + (d_3^2 + d_3d_1 + d_1^2)\frac{t^3}{6} + \ldots ]}
              {(d_3 - d_1)(d_2 - d_1)} \right.\\
      &\phantom{= P_2 P_3 \left [ \right .}
              \left .\quad+\frac{(d_3 - d_2) [t - (d_3 + d_2)\frac{t^2}{2} + (d_3^2 + d_3d_2 + d_2^2)\frac{t^3}{6} + \ldots ]}
              {(d_3 - d_2)(d_1 - d_2)} \right ]\\
      &= P_2 P_3 \left [ \frac{(d_2 - d_1)\frac{t^2}{2} + [d_3(d_1-d_2) + (d_1^2-d_2^2)]\frac{t^3}{6} + \ldots }
              {d_2 - d_1} \right ]\\
      &= P_2 P_3 \left [ \frac{t^2}{2} - (d_3 + d_2 + d_1) \frac{t^3}{6} + \ldots \right]\\
      &= P_2 P_3\;t^2 \left [ \frac{1}{2} - \frac{t}{6} (d_3 + d_2 + d_1) + \ldots \right]
    \end{split}
  \end{align}
  which has the form of Equation \ref{eqn:math.expansion.final}.
  
  Whether in the Laplace transform domain or the time domain, there is
  a necessity to calculate coefficients of the form:
  \begin{equation}
    \{c_i\} = \left \{\sum_{j=1}^N d_j \ ,\ \sum_{j=1}^N d_j \sum_{k=j}^N
              d_k\ ,\ \sum_{j=1}^N d_j \sum_{k=j}^N d_k \sum_{l=k}^N d_l\ ,\ \ldots\
              \right \}.
  \end{equation}
  
  A different form for these coefficients becomes apparent when $N=2$
  or $N=3$.  The coefficients, $\{c_i\}$, are:
  \begin{equation}
    \{c_i\} = \left \{ d_1 + d_2\ ,\ d_1(d_1+d_2)+d_2^2\ ,
              \ d_1\left[d_1(d_1+d_2)+d_2^2\right] + d_2^3\ , \ldots \right\}
  \end{equation}
  or
  \begin{equation}
    \begin{split}
      \{c_i\} &= \left \{ d_1+d_2+d_3,\ d_1(d_1+d_2+d_3)+d_2(d_2+d_3)+d_3^2,\ \right. \\
              &\quad\quad \left. d_1\left[d_1(d_1+d_2+d_3)+d_2(d_2+d_3)+d_3^2 \right] 
               + d_2\left[d_2(d_2+d_3)+d_3^2\right] +d_3^3\ , \ldots \right \}.
    \end{split}
  \end{equation}
  This shows the following pattern, assuming $\{\lambda_{0,j}\} = 1;\ j=[1,N]$:
  \begin{align}
    \lambda_{ij} &= \sum_{k=j}^N d_k \lambda_{i-1,k}\\
    c_i &= \lambda_{i1}.
  \end{align}
  
  This last form leads to an efficient way to calculate these
  coefficients using matrix multiplications.  If we form a matrix,
  $M$, with elements $m_{ij} = d_j; j \geq i$:
  \begin{equation}
    \mat{M} = \begin{bmatrix}
      d_1 & d_2 & d_3 & \ldots & d_N\\
      0 & d_2 & d_3 & \ldots & d_N\\
      0 & 0 & d_3 & \ldots & d_N\\
      \vdots & & & \ddots & \vdots\\
      0 & 0 & 0 & \ldots & d_N\\
    \end{bmatrix},
  \end{equation}
  it is clear that $\lambda_1 = [\mat{M}\cdot\vec{1}]$ and that
  $\lambda_i = [\mat{M}^i\cdot\vec{1}]$.  Therefore,
  \begin{equation}
    c_i = \lambda_{i1} = \left [M^i\cdot \vec{1}\right]_1.
  \end{equation}

  Since the direct calculation of
  \begin{equation}
    \sum_{j_1=1}^N d_{j_1} \sum_{j_{2}=j_1}^N d_{j_{2}}
        \sum_{j_{3}=j_2}^N d_{j_{3}} \cdots \sum_{j_{n-1}=j_{n-2}}^N
        d_{j_{n-1}}\sum_{j_n=j_{n-1}}^N d_{j_n} =
        \prod_{l=n}^1\sum_{j_l=j_{l-1}}^Nd_{j_l}
  \end{equation}
  tends to require $O(N^n)$ calculations, the matrix method above will
  be highly advantageous since it requires only $O(nN^3)$
  calculations.
  
  Once these coefficients have been calculated, they are then used to
  calculate the time response using Equation
  \ref{eqn:math.expansion.final}:
  \begin{equation}
    \begin{split}
      f(t) &= t^{n} \left[ \frac{1}{n!} - \frac{t}{(n+1)!}\sum_{l=j}^i d_l 
                           +\frac{t^2}{(n+2)!}\sum_{l=j}^{i}d_l \sum_{k=l}^i d_k\right.\\  
           &\phantom{= t^{n-1} \left[\right.}
                    \left.\quad\quad-\frac{t^3}{(n+3)!}\sum_{l=j}^{i}d_l \sum_{k=l}^i d_k \sum_{m=k}^i d_m 
                          +\cdots \right ].\\
    \end{split}\tag{\ref{eqn:math.expansion.final}}
  \end{equation}

\end{chapter}


\addtocounter{tocdepth}{-1}
\begin{chapter}{Users' Guide\label{app:user.guide}}
  
  The usage of ALARA is fairly straightforward, requiring little
  knowledge of the code's inner workings.  Of course, to ensure that
  ALARA is well-suited to the problems that you are trying to solve,
  you are encouraged to read chapters \ref{chap:physical} and
  \ref{chap:math} and understand the physical and mathematical
  modeling characteristics of ALARA.
  
  This chapter will describe the command-line options of ALARA and
  then describe the basic support files necessary to run ALARA.
  
  \begin{section}{Command-line Options}\label{app:user.cmd}
    
    \begin{center}
      \textbf{\texttt{alara [-t }\textsl{tree\_output\_filename}\texttt{]
          [-h] [-v }\textsl{[n]}\texttt{]} \textsl{input\_filename}}
    \end{center}
    ALARA currently supports 4 command-line options:
    \begin{description}
    \item[\texttt{-h} Help]\ \\
      This option will print a short help message describing the
      command-line options and their usage.
    \item[\texttt{-t }\textsl{tree\_output\_filename} Tree File]\ \\
      This option allows you to define the file name for the tree file
      containing tree creation and truncation information.  This
      information can later be used for basic pathway analysis.  The
      default is to create no tree file.  The format of this file is
      described in section \ref{app:user.output.tree}.
    \item[\texttt{-v }\textsl{[n]} Verbose]\ \\
      This option alters the output verbosity.  Without this option,
      only the final results will be displayed.  By using this option,
      details of the calculation are included in the output.  The
      level of detail is controlled by the optional value, \textsl{n},
      having a value between 1 (least detail) and 7 (most detail).  If
      no value is given, it defaults to 1.
    \item[\textsl{input\_filename} Input File]\ \\
      This option allows you to define the input filename to be used
      by ALARA.  If no name is specified, the input will be read from
      \texttt{stdin}.
    \end{description}
    
    Output from \ALARA\ is written to the \texttt{stdout} stream.  To
    capture the output in a file, simply use the standard method of
    your operating system.
    
  \end{section}
  
  
  \begin{section}{Input File Description\label{app:user.input}}
    
    The input file for ALARA has been designed to ensure that the
    input information is easy to understand, edit and comment.  This
    is possible by using a very free format permitting comments, blank
    lines, inclusion of other files, and arbitrary ordering of the
    input information.  After reading the full input file, \ALARA\ 
    performs various cross-checks and cross-references to ensure that
    the input data is self-consistent.  It then goes on to pre-process
    the data for the calculation.  Every attempt has been made to give
    useful error messages when the data is not consistent.

    \begin{subsection}{General}
    
      There are 19 possible input block types.  These blocks can
      appear in any order and many blocks can occur more than once, if
      needed.  One block type, \texttt{convert\_lib}, is only used to
      convert data library formats, and will cause \ALARA\ to halt if
      used in an input file.  The other 18 input blocks are:

      {
        \renewcommand{\baselinestretch}{1}\normalsize
        \noindent\begin{tabular}{p{2in}p{2in}p{2in}}
          \begin{enumerate}
          \item \texttt{geometry}
          \item \texttt{dimension}
          \item \texttt{major\_radius}
          \item \texttt{minor\_radius}
          \item \texttt{volumes}
          \item \texttt{mat\_loading}
          \end{enumerate} &
          \begin{enumerate}\setcounter{enumi}{6}
          \item \texttt{mixture}
          \item \texttt{flux}
          \item \texttt{spatial\_norm}
          \item \texttt{schedule}
          \item \texttt{pulsehistory}
          \item \texttt{truncation}
          \end{enumerate} &
          \begin{enumerate}\setcounter{enumi}{12}
          \item \texttt{output}
          \item \texttt{cooling}
          \item \texttt{material\_lib}
          \item \texttt{element\_lib}
          \item \texttt{data\_library}
          \item \texttt{dump\_file}
          \end{enumerate}
        \end{tabular}
        }

      Not all input blocks are required, with some having default
      values and others being unnecessary for certain problems.  There
      are also some input blocks which are incompatible with each
      other.  While superfluous input blocks may go unnoticed (there
      are occasional warnings), incompatible input blocks will create
      an error.
      
      Most input blocks allow the user to define their own symbolic
      names for cross-referencing the various input data.  Any string
      of characters can be used as long as its does not contain any
      whitespace (spaces, tabs, new-lines, etc.).  It is considered
      dangerous, however, to use a keyword as a symbolic name.  If the
      input file is correct, it will function properly, but if there
      are errors in the input file, the usage of keywords as symbolic
      names may make the error message irrelevant.  The keywords
      include those listed in the above list and the keyword ``end''.
      While many input blocks of fixed length require nothing to
      indicate the end of the block, some blocks have a variable
      length and require the keyword ``end'' to terminate the block.
      
      Some input elements represent times and can be defined in a
      number of different units.  When this is the case, the floating
      point time value should be followed by a single character
      representing the following units:

      \begin{center}
        \renewcommand{\baselinestretch}{1}\normalsize
        \begin{tabular}{|l|l|l|}
          \hline
          \texttt{[s]econd} & 
          \texttt{[m]inute} = 60 seconds& 
          \texttt{[h]our} = 60 minutes\\\hline
          \texttt{[d]ay} = 24 hours& 
          \texttt{[w]eek} = 7 days& 
          \texttt{[y]ear} = 52 weeks\\\hline
          \multicolumn{3}{|c|}{\texttt{[c]entury} = 100 years}\\\hline
        \end{tabular}
      \end{center}

      One input file can be included in another with the
      \texttt{\#include} directive, similar to the C programming
      language.  Any number of files can be included, and included
      files can also contain directives to include other files.  The
      only restriction is that the inclusion must not occur within an
      input block!
      
      All other lines in which the first non-space character is the
      pound sign (or number sign) (\#) are considered as comments.
      Comments can also be used after any single word input (an input
      value with no whitespace) by using the same comment character
      (\#).  Such comments extend to the end of the current line.
      Blank lines are permitted anywhere in the input file.
      
      When length units are implied in the input for sizes and
      dimensions, it is only important that all implied units be
      consistent but not what unit is implied.
    \end{subsection}
    
    \begin{subsection}{\texttt{geometry}\label{app:user.input.geom}}
    
      This input block is only necessary when defining a geometry
      using the \texttt{dimension} input block, but may always be
      included.  It should only occur once.  This input block takes a
      single argument which must be one of the following:
      \begin{center}
        \texttt{point | rectangular | cylindrical | spherical | torus}
      \end{center}
      This input block should not be terminated.

      \begin{center}
        \renewcommand{\baselinestretch}{1}\normalsize
        \begin{boxedverbatim}
# problem geometry
geometry spherical          
\end{boxedverbatim}
      \end{center}

      If using the \texttt{dimension} input block to define the
      geometry and the type is \texttt{torus}, the
      \texttt{major\_radius} input block is required and the
      \texttt{minor\_radius} block may also be required.
    \end{subsection}

    \begin{subsection}{\texttt{dimension}\label{app:user.input.dim}}
      
      This input block is used to define the geometry layout, and
      should be included once for each dimension needed in the
      problem.  The dimension block's first element indicates which
      dimension is being defined and should be one of the following:
      \begin{center}
        \texttt{ x | y | z | r | theta | phi }
      \end{center}
      \ALARA\ will check to ensure that only dimensions relevant to
      the defined geometry are included.  For example, defining the
      '\texttt{x}' dimension in a spherical problem will generate an
      error.
      
      The dimension block's next element is the first zone's lower
      boundary, expressed as a floating point number.  This is
      followed by a list of pairs, one pair for each zone: an integer
      specifying the number of intervals in this zone in this
      dimension and a floating point number indicating the zone's
      upper boundary. This list is terminated with the \texttt{end}
      keyword.
      
      \begin{center}
        \renewcommand{\baselinestretch}{1}\normalsize
        \begin{boxedverbatim}
# sample dimension for spherical problem 
dimension r 1.0      # inside radius 
         10 2.0      # 10 intervals in the first zone (1.0,2.0)
         15 3.0      # 15 intervals in the next zone (2.0,3.0) 
end
\end{boxedverbatim}
      \end{center}
      
      Since this method of defining the geometry calculates the the
      fine mesh intervals' zone membership and volume from the
      \texttt{dimension} data, it is incompatible with the
      \texttt{volumes} input block.  Including both will generate an
      error message.
    \end{subsection}

    \begin{subsection}{\texttt{major\_radius} and  \texttt{minor\_radius} \label{app:user.input.tor_radii}}
      
      These two input blocks are used to define the major and minor
      radii of toroidal geometries.  They are only needed if defining
      a toroidal geometry with \texttt{dimension} input blocks, and
      each should only be included once.  Furthermore, if the minor
      radius dimension is defined with a \texttt{dimension} block, the
      \texttt{minor\_radius} input block is not required.  In both
      cases, these input blocks have a fixed size, with a single
      argument specifying the radius as a floating point number.

      \begin{center}
        \renewcommand{\baselinestretch}{1}\normalsize
        \begin{boxedverbatim}
# major and minor radii of torus
major_radius     2.0
minor_radius     0.5          
\end{boxedverbatim}
      \end{center}

    \end{subsection}

    \begin{subsection}{\texttt{volumes}\label{app:user.input.vol}}
      
      This input block is used to define the fine mesh intervals'
      volumes and zone membership.  This block can be used instead of
      the \texttt{dimension} method of defining the geometry.  If both
      are used, an error will result.  This block should only occur
      once.  Multiple occurrences will result in undefined behavior.
      
      This input block should be a list of pairs, one pair for each
      interval.  Each pair consists of a floating point value for the
      volume of that interval and the symbolic name of the zone
      containing that interval.  These symbolic names should
      correspond with the symbolic names given to the zones in the
      \texttt{mat\_loading} input block (see section
      \ref{app:user.input.loading}).  This list must be terminated
      with the keyword \texttt{end}.

      \begin{center}
        \renewcommand{\baselinestretch}{1}\normalsize
        \begin{boxedverbatim}
# list of fine mesh intervals
volumes
   0.5     vacuum_vessel
   1.5     shield_zone
   2.34    blanket_zone
   1.92    first_wall
end          
\end{boxedverbatim}
      \end{center}
    \end{subsection}

    \begin{subsection}{\texttt{mat\_loading}\label{app:user.input.loading}}
      
      This input block is used to indicate which mixtures are
      contained in each zone.  This block is a list with one pair of
      entries for every zone.  Each pair consists of a symbolic name
      for the zone and a symbolic name for the mixture contained in
      that zone.  This list is terminated by the keyword \texttt{end}.
      This block should only occur once.  Multiple occurrences will
      result in undefined behaviour.

      If the geometry is defined using the \texttt{dimension} input
      blocks, there number of zones defined here must match the number
      of zones defined in the \texttt{dimension} blocks exactly; if
      not, an error results.  If the \texttt{volumes} method is used
      to define the geometry, this block uniquely determines the
      number of zones.  The symbolic name for the mixture must match
      one of the \texttt{mixture} definitions exactly, or be the
      keyword '\texttt{void}', indicating that this zone is empty of
      material.

      \begin{center}
        \renewcommand{\baselinestretch}{1}\normalsize
        \begin{boxedverbatim}
# material loadings for all zones
mat_loading
   vacuum_vessel  VV_materials
   shield_zone    shield_mixture
   blanket_zone   breeding_blanket
   first_wall     liquid_FW
   scrapeoff      void          
end
\end{boxedverbatim}
      \end{center}
      
    \end{subsection}
 

    \begin{subsection}{\texttt{mixture}\label{app:user.input.mix}}
      
      This kind of block is used to define the composition of a
      mixture.  This block can occur as many times as necessary to
      define all the mixture compositions in the problem.  Any
      mixtures that are defined, but not used in the problem will
      generate a warning and be removed from the list of mixtures.
      
      The first element of a \texttt{mixture} block is the symbolic
      name used to refer to this mixture elsewhere in the input file.
      Following this is a list of triplets with one triplet for each
      mixture component.  The list must be terminated with the keyword
      '\texttt{end}'.  The first element of each triplet describes the
      type of that component and should be one of:
      \begin{center}
        \texttt{material | element | isotope | like | target}
      \end{center}
      
      The remaining elements in each triplet are interpreted as
      follows, based on this first element:
      \begin{description}
      \item[\texttt{material}] The second element in this triplet is
        the symbolic name of a material definition existing in the
        material library (see section \ref{app:user.matlib}).  The
        final element is a floating point value representing the
        relative density of this material.  This value, usually
        between 0 and 1, will be multiplied by the density found in
        the material library to define the density of this component.
        It can also be interpreted as the volume fraction of this
        material in this mixture.
      \item[\texttt{element}] The second element in this triplet is
        the element's chemical symbol.  This element will be expanded
        into a list of isotopes using the natural isotopic abundances
        found in the element library (see section
        \ref{app:user.elelib}).  The final element is a floating point
        value representing the relative density of this material.
        This value, usually between 0 and 1, will be multiplied by the
        standard theoretical density found in the element library to
        define the density of this component.  It can also be
        interpreted as the volume fraction of this element in this
        mixture.
      \item[\texttt{isotope}] The second element in this triplet is a
        symbolic name for the isotope in the format ZZ-AAA, where ZZ
        is the chemical symbol and AAA is the mass number, for
        example, \texttt{i-127, ca-40} or \texttt{pb-207}.  The final
        element of this triplet is the number density of this isotope
        in the mixture.
      \item[\texttt{like}] This type of entry is provided as a
        convenience and indicates that this component is like another
        user-defined mixture, with a potentially different density.
        The second element of this triplet is the symbolic name of
        another mixture definition.  If the other mixture definition
        is not found, an error will result.  The triplet's final
        element is a relative density, used to normalize the density
        as defined in that mixture's own definition.  This might be
        used when a user-defined mixture makes up part of another
        mixture.  [Hint: it is permissible to define a mixture that is
        not used in any zones, but only used as part of another
        mixture.]
      \item[\texttt{target}] This type of entry is used to initiate a
        reverse calculation (see section \ref{sec:physical.reverse})
        and define the target isotopes for the reverse calculation.
        The user can define an arbitrary number of target isotopes.
        The second element of this triplet is one of the keywords
        \texttt{element} or \texttt{isotope}, indicating what kind of
        target this is.  The final element is the symbolic name of
        either the \texttt{element} or \texttt{isotope}, as defined in
        the corresponding entries in this list above.  There is no
        density in this type of entry.  If a target is of type
        \texttt{element}, the element will be expanded using the
        element library to create a list of isotopes.
      \end{description}
 
      \begin{center}
        \renewcommand{\baselinestretch}{1}\normalsize
        \begin{boxedverbatim}
# definition of vacuum vessel
mixture VV_materials
   material SS316-L 1.0
end

# definition of fusion shield
mixture shield_mixture
   like     vacuum_vessel  0.6
   element  he             0.4
end

# definition of fusion breeding blanket
mixture breeding_blanket
   element  li      0.95
   isotope  li-6    6.02e23
   target   isotope h-3
end
\end{boxedverbatim}
      \end{center}

      Note that even if a target is defined in only one mixture, it
      will cause the whole problem to be run as a reverse problem.
      There is therefore little purpose in having mixture definitions
      without targets (such as in this example).
    \end{subsection}

    \begin{subsection}{\texttt{flux}\label{app:user.input.flux}}
      
      This input block defines a set of flux spectra.  Since different
      parts of the irradiation history can have different flux
      spectra, this block may occur as many times as necessary to
      represent all the different necessary flux definitions.  The
      first element of this block is a symbolic name, used to refer to
      this flux spectra definition.  The other elements of this block
      are a filename, a floating point scalar normalization, an
      integer skip value (see below), and flux type indicator string,
      respectively.

      The flux filename should indicate which file contains this flux
      information, including path information appropriate to find the
      file from the directory in which \ALARA\ will be run.
      
      The scalar normalization permits uniform flux scaling at all
      spatial points (as opposed to the \texttt{spatial\_norm}
      information in the next section).  All groups of all fluxes in
      this definition will be multiplied by this value.
      
      The skip value indicates how many N-group flux entries to skip
      in this file before reading the first flux.  This permits the
      user to have one file with many different flux spectra.  For
      example, if the schedule requires two different flux spectra for
      N different fine mesh points, the data for the first one may be
      at the beginning of the file, with a skip of 0, while the data
      for the second flux definition would be after these first
      fluxes, with a skip of N.
      
      The last element is a character string indicating the flux
      file's format.  Currently the only supported format is
      \texttt{default}.  The default flux file format consists of one
      list of group fluxes per spatial point.  There are no other
      entries and this can be freely formatted, although comments are
      not permitted.

      \begin{center}
        \renewcommand{\baselinestretch}{1}\normalsize
        \begin{boxedverbatim}
# flux for first part of irradiation schedule
flux high_yield_flux ../n.transport/machine.flux 1.0 0 default

# flux for second part of irradiation schedule
flux low_yield_flux ../n.transport/machine.flux 1.0 432 default

# flux for final part of irradiation schedule
flux attenuated_high_yield ../n.transport/machine.flux 0.5 0 default
\end{boxedverbatim}
      \end{center}
      
      [Hint: Different flux definitions might use exactly the same
      flux values (same flux file and skip value) but a different
      scaling value.]
    \end{subsection}

    \begin{subsection}{\texttt{spatial\_norm}\label{app:user.input.norm}}
      
      This input block allows the user to specify a scalar flux
      normalization for each fine mesh interval, such as might be
      required to re-normalize the results of a transport calculation
      on an approximated geometry.  The number of normalizations must
      be at least as many as the number of defined intervals,
      regardless of how the intervals are defined (\texttt{dimension}
      vs. \texttt{volumes}).  If there are too few, an error will
      result; if there are too many, a warning will result.
      
      This block consists of a list of floating point normalization
      values, one value for each interval, and requires the
      \texttt{end} keyword to terminate the list.

      \begin{center}
        \renewcommand{\baselinestretch}{1}\normalsize
        \begin{boxedverbatim}
# normalizations to convert cylindrical model
# to toroidal equivalent
spatial_norm
   0.8
   0.85
   0.9
   1.0
   1.03
   1.08
   1.1
end
\end{boxedverbatim}
      \end{center}

      [Hint: if these values are purely a function of problem
      geometry, and not mixture composition, it is possible that many
      problems have the same spatial normalization.  Put this data in
      a separate file and \texttt{\#include} it when you need it.]
      
    \end{subsection}
   
    \begin{subsection}{\texttt{schedule}\label{app:user.input.sched}}
      
      This kind of block is used to define a single schedule in the
      full irradiation history hierarchy.  Since the hierarchy may be
      composed of many schedules, this block might occur many times.
      The first element in this input block is a symbolic name by
      which this schedule can be referred to.  Following this is a list
      of items occurring in this schedule.  There are two possible
      types for each item, and their may be an arbitrary list of items
      in a schedule.  This list must be terminated with the keyword
      '\texttt{end}'.  See section \ref{app:user.schedules} for more
      information about defining schedules.
      
      The first type of item is a simple pulse and the entries for
      this kind of item are a floating point operating time, a single
      character defining the units of that operating time, a symbolic
      flux name, a symbolic pulsing definition name, a floating point
      post-item delay time, and a single character defining the units
      of that delay time.
      
      The second type of item is a sub-schedule and the entries for
      this kind of item are a symbolic name for the sub-schedule, a
      symbolic pulsing definition name, a floating point post-item delay
      time, and a single character defining the units of that delay
      time.
      
      In both cases, if the symbolically named items (flux, pulsing
      definition, or schedule) are not found during cross-referencing,
      an error results.

      \begin{center}
        \renewcommand{\baselinestretch}{1}\normalsize
        \begin{boxedverbatim}
# top level schedule
schedule top
    # a schedule of operation defined by 'phase_1_sched' 
    #     repeated with a cycle defined by 'phase_1_cycle'
    #     after which there is a 10 week delay
  phase_1_sched phase_1_cycle 10 w
    # a schedule of operation defined by 'phase_2_sched' 
    #     repeated with a cycle defined by 'phase_2_cycle'
    #     after which there is a 5 week delay
  phase_2_sched phase_2_cycle 5 w
end

# phase 1 schedule
schedule phase_1_sched
    # a pulsing regimen with 1800 s pulses at flux 'high_yield_flux' 
    # pulsed with '5_week_plan_short' followed by a 1 week delay
  1800 s high_yield_flux 5_week_plan_short 1 w
    # a special schedule for cleaning the facility, 'cleaning_sched'
    # with a pulsing history 'cleaning_cycle' followed by no delay
  cleaning_sched cleaning_cycle 0 s
end

# cleaning sched
schedule cleaning_sched
    # two consecutive pulsing regimen, each with one day pulses and
    # pulsing history 'daily_week_long' followed by a 1 day delay
    # the first uses 'low_yield_flux' and the other 'low_energy_flux'
  1 d low_yield_flux daily_week_long 1 d
  1 d low_energy_flux daily_week_long 1 d
end

# phase 2 schedule
schedule phase_2_sched
    # a pulsing regimen with 1 hour pulses at flux 
    # 'high_yield_flux' pulsed with '5_week_plan_long' after which
    # there is a 1 week delay
  1 h high_yield_flux 5_week_plan_long 1 w
    # same cleaning phase as in phase 1 schedule
  cleaning_sched cleaning_cycle 0 s
end
\end{boxedverbatim}
      \end{center}

    \end{subsection}

    \begin{subsection}{\texttt{pulsehistory}\label{app:user.input.pulse}}
      
      This kind of input block defines the multi-level pulsing
      histories referenced in the \texttt{schedule} definitions.  See
      section \ref{app:user.schedules} for more information about
      defining schedules and histories.  Since many different pulsing
      histories may be used throughout the hierarchy of schedules,
      this block may occur many times.

      The first element of each block is a symbolic name for referring
      to this pulsing schedule.  Following this is a list of pulsing
      level definition triplets, each consisting of an integer number
      of pulses, a floating point delay time between pulses, and a
      single character defining the units of that delay time.  Since
      an arbitrary number of pulsing levels is allowed, this list must
      be terminated with the keyword '\texttt{end}'.

      \begin{center}
        \renewcommand{\baselinestretch}{1}\normalsize
        \begin{boxedverbatim}
# define 5 weeks of the short pulses (1800 s = 1/2 hour)
pulsehistory 5_week_plan_short
   4  90 m   # 4 pulses each day with one every 2 hours
   5  17.5 h # 5 days with 16 hours + 90 minutes delay (overnight)
   5  2.73 d # 5 weeks with 2 days + 17.5 hours delay (weekends)
end

# define 5 weeks of the long pulses (1 h)
pulsehistory 5_week_plan_long
   4  1 h    # 4 pulses each day with one every 2 hours
   5  17 h   # 5 days with 16 hours + 1 h delay (overnight)
   5  2.71 d # 5 weeks with 2 days + 17 hours delay (weekends)
end
\end{boxedverbatim}
      \end{center}

    \end{subsection}

    \begin{subsection}{\texttt{truncation}\label{app:user.input.trunc}}
      
      This fixed sized input block defines the parameters used in
      truncating the activation trees.  See section
      \ref{sec:physical.chains.trunc} for a detailed discussion of the
      tree truncation issue.  The first element of this block is the
      truncation tolerance and the second is an ignore tolerance.
      When testing the relative atom loss (or relative production in
      reverse calculations), any value higher than the truncation
      tolerance will result in continuing the tree while lower values
      will result in truncation.  If the value is also lower than the
      ignore tolerance, that node is completely ignored.

      \begin{center}
        \renewcommand{\baselinestretch}{1}\normalsize
        \begin{boxedverbatim}
# defined a 1/10000 tolerance, ignoring 1e-10
truncation   1e-4    1e-10
\end{boxedverbatim}
      \end{center}
      
    \end{subsection}

    \begin{subsection}{\texttt{output}\label{app:user.input.output}}
      
      This kind of input block allows the user to define the output's
      resolution and format.  The first element of an output format
      block indicates the resolution and should be one of:
      \begin{center}
        \texttt{ interval | zone | mixture }
      \end{center}
      This is followed by a list of output types and modifiers
      described in the following table:

      \begin{center}
        \renewcommand{\baselinestretch}{1}\normalsize
        \begin{tabular}{|c|l|}
          \hline
          keyword & function \\\hline\hline
          component & component breakdown in addition to total response\\\hline
          number\_density & number density result of all produced isotopes\\\hline
          specific\_activity & specific activity of all radioactive isotopes\\\hline
          total\_heat & total decay heat\\\hline
          alpha\_heat & total alpha heating\\\hline
          beta\_heat & total beta heating\\\hline
          gamma\_heat & total gamma heating\\\hline
        \end{tabular}

        \vspace{2\baselineskip}

        \begin{boxedverbatim}
# output only total zone number densities (no component breakdown)
output zone
   number_density
end

# now output activities and decay heats for zones
#   with component breakdown
output zone
   component
   specific_activity
   total_heat
end

# now output total activities for mixtures
output mixture
   specific_activity
end
\end{boxedverbatim}
      \end{center}      

      See section \ref{app:user.output} for information on
      interpreting the output files generated by \ALARA.
    \end{subsection}

    \begin{subsection}{\texttt{cooling}\label{app:user.input.cool}}
      
      This input block is used to define the after-shutdown cooling
      times at which the problem will be solved.  Multiple occurrences
      will result in undefined behavior.  This block is simply a list
      of times, where each time consists of a floating point time
      followed by a single character defining the time's units.  Since
      an arbitrary number of cooling times can be solved, this list
      must be terminated with the keyword '\texttt{end}'.

      \begin{center}
        \renewcommand{\baselinestretch}{1}\normalsize
        \begin{boxedverbatim}
# a wide array of cooling times
cooling
     1 m
     1 d    
     1 w
   0.5 y
     1 y
    10 y
     1 c
end          
\end{boxedverbatim}
      \end{center}

    \end{subsection}

    \begin{subsection}{\texttt{material\_lib} and \texttt{element\_lib}\label{app:user.input.matlibs}}
      
      These two input blocks are used to specify the libraries to be
      used for looking up the definitions of materials and elements
      when they are given as mixture components (see section
      \ref{app:user.input.mix}).  Each block has a single element
      consisting of the filename to be used in each case, including
      appropriate path information to find that file from the
      directory where \ALARA\ is being run.  For more information on
      the format of these libraries, see section
      \ref{app:mat_el_libs}.

      \begin{center}
        \renewcommand{\baselinestretch}{1}\normalsize
        \begin{boxedverbatim}
material_lib  /alara/data/matlib/magnetic_fusion
element_lib   /alara/data/std.elelib          
\end{boxedverbatim}
      \end{center}
    \end{subsection}

    \begin{subsection}{\texttt{data\_library}\label{sec:user.input.datalib}}
      
      This input block is used to define the type and location of the
      nuclear data library.  The first element of this block is
      character string defining the type of library.  The subsequent
      elements indicate the file's location.  Currently accepted
      library types are:
      \begin{description}
      \item[alaralib] Standard \ALARA\ v.2 binary library.  This
        library type requires a single filename indicating the
        library's location.
      \item[adjlib] Standard \ALARA\ v.2 reverse library  This
        library type requires a single filename indicating the
        library's location.
      \item[eaflib] Data library following EAF conventions (ENDF/B).
        This library type requires two filenames, the transmutation
        library and the decay library, respectively.  These libraries
        will be read and processed, creating an \ALARA\ v.2 binary
        library with the name '\texttt{alarabin}' for use in
        subsequent calculations.  Alternatively, this library could be
        converted to an \ALARA\ v.2 binary library as a separate
        process using the \texttt{convert\_lib} function.
      \end{description}

      For both types of \ALARA\ v.2 library, the extension ``.lib''
      will be added to the filename indicated in this input block.
      Otherwise, all filenames should include appropriate path
      information to find the file from the directory in which \ALARA\
      will be run.

      \begin{center}
        \renewcommand{\baselinestretch}{1}\normalsize
        \begin{boxedverbatim}
# convert and use an EAF formated FENDL2 library
data_library eaflib /alara/data.src/FENDL2/fendlg-2.0
       /alara/data.src/FENDL2/fendld-2.0          
\end{boxedverbatim}
      \end{center}

    \end{subsection}

    \begin{subsection}{\texttt{dump\_file}\label{sec:user.input.dump}}
      
      This input block defines the filename to use for the binary data
      dump produced during a run of \ALARA.  This is currently used to
      store the intermediate results during the calculation, and will
      be extended in the future to allow sophisticated post-processing
      of the data.  This filename should be a valid name for a new
      file, including path information appropriate for the directory
      where \ALARA\ will be run.  Note that if the dump file already
      exists, it will be overwritten with no warning.  If this input
      block is omitted, the default name '\texttt{alara.dump}' will be
      used.

      \begin{center}
        \renewcommand{\baselinestretch}{1}\normalsize
        \begin{boxedverbatim}
# define a dump file name
dump_file testing/test_problem.dump          
\end{boxedverbatim}
      \end{center}

    \end{subsection}

  \end{section}
  
  \begin{section}{Defining Irradiation Schedules\label{app:user.schedules}}
    
    With the added flexibility in irradiation schedule definition
    comes added complication.  All attempts have been made for this to
    be a straightforward process, and this section should help make it
    clearer.
    
    Since the hierarchical systems of schedules are based on
    repetition, the simplest way to develop the input representing any
    given irradiation schedule is to identify repeated blocks within
    the schedule (or some portion).  Each of these blocks is then
    given the same treatment, with looking for repetition within each
    block and so on, until the last stage, where the repeated element
    is a single pulse.

  \end{section}

  \begin{section}{\ALARA\ Output File Formats\label{app:user.output}}

    \begin{subsection}{Output File}
      
      As described in section \ref{app:user.cmd} on command-line
      arguments, various levels of output are available during the
      calculation.  The part of the output file will contain
      this verbose output, including confirmation of the input data
      and details of the cross-referencing and preprocessing of the
      input.
      
      The second part of the output file shows details on the tree
      building process, ranging from a simple list of the root
      isotopes being solved and statistics on the size and speed of
      the solution, to details on the chain growth and truncation
      calculations (depending on the verbosity specified on the
      command-line).
      
      \begin{figure}[htbp]
        \begin{center}
          \epsfig{file=eps/out_fmt.eps}
        \caption{Output file structure.}
        \label{fig:output.desc}
        \end{center}
      \end{figure}
    
      The final part of the output file are the results, as requested
      by the user in the input file.  This output will include one
      section for each output format description given by the user.
      Each of these sections will be divided into blocks as shown in
      figure \ref{fig:output.desc}.  There are two types of tables.
      The first type has a row for each isotope produced in the
      problem that has a non-zero response.  For example, the specific
      activity in a water filled zone of the benchmark problem of
      chapter \ref{chap:valid} might be:
      \begin{center}
        \renewcommand{\baselinestretch}{1}\normalsize
        \begin{boxedverbatim}
isotope  shutdown         1 h         1 d        30 d         1 y   
====================================================================
h-3     1.9419e+08  1.9419e+08  1.9416e+08  1.9330e+08  1.8361e+08  
c-14    1.3073e+08  1.3073e+08  1.3073e+08  1.3073e+08  1.3072e+08  
c-15    3.9876e+10  0.0000e+00  0.0000e+00  0.0000e+00  0.0000e+00  
n-16    7.9376e+13  0.0000e+00  0.0000e+00  0.0000e+00  0.0000e+00  
n-17    2.3761e+10  0.0000e+00  0.0000e+00  0.0000e+00  0.0000e+00  
n-18    1.5607e+09  0.0000e+00  0.0000e+00  0.0000e+00  0.0000e+00  
o-19    1.7727e+09  0.0000e+00  0.0000e+00  0.0000e+00  0.0000e+00  
====================================================================
total   7.9443e+13  3.2492e+08  3.2490e+08  3.2403e+08  3.1433e+08  
\end{boxedverbatim}
      \end{center}
      The second type of table has a row for each point in the
      requested resolution, giving the total response at that point.
      The specific activity in all the benchmark problem's zones might
      be:
      \begin{center}
        \renewcommand{\baselinestretch}{1}\normalsize
        \begin{boxedverbatim}
Totals for all zones.
zone     shutdown         1 h         1 y   
============================================
1       central_zone (void)                 
2       1.8628e+09  1.3445e+09  5.3928e+07  magnet_coil (TF_Coil)
3       6.7569e+08  2.1433e+08  2.1405e+06  magnet_ins (Insulator)
4       gap1 (void)                         
5       2.8115e+09  1.4657e+08  4.2568e+07  i_VV_shield1 (Pb_B4C)
6       3.7343e+10  3.3576e+10  3.5317e+09  i_VV_wall1 (Inconel)
.
.
.
25      2.0427e+16  1.4648e+16  1.0754e+14  i_fw_cube (Cu-Be-Ni)
26      5.8139e+15  9.3164e+13  8.8089e+13  i_fw_Be (Be)
27      i_Scrape (void)                    
28      plasma (void)                       
29      o_scrape (void)                   
30      7.6731e+15  1.2204e+14  1.1539e+14  o_fw_Be (Be)
31      2.3041e+16  1.6188e+16  1.4785e+14  o_fw_cube (Cu-Be-Ni)
.
.
.
45      2.3003e+11  2.8462e+06  2.7750e+06  o_b_water1 (water)
46      9.6034e+12  8.1892e+12  1.2897e+12  o_b_steel1 (steel)
47      gap3 (void)                         
48      6.5203e+12  5.7575e+12  6.0297e+11  o_VV_wall2 (Inconel)
49      4.3277e+11  3.6355e+11  4.6945e+10  o_VV_shield2 (Steel_Water)
50      5.3731e+09  4.8643e+09  5.1452e+08  o_VV_wall1 (Inconel)
51      4.1986e+08  2.0902e+07  6.2052e+06  o_VV_shield1 (Pb_B4C)
============================================
\end{boxedverbatim}
      \end{center}

      If this is a reverse calculation, the entire structure defined
      above will be repeated for each target isotope.
    \end{subsection}

    \begin{subsection}{Tree File\label{app:user.output.tree}}
      
      ALARA also produces a so-called tree file to allow some
      rudimentary pathway analysis (see Section \ref{app:user.cmd}).
      The tree file contains much information about the creation and
      truncation of the trees and chains used to calculate the
      transmutation and activation in the problem.
      
      One tree will be created for each initial isotope.  All the
      information given for this isotope is based on the flux chosen
      for the truncation calculations of this isotope, namely, the
      group-wise maximum flux across all the intervals in which the
      initial isotope exists.  An entry for an isotope in the tree
      will look like this:
      \begin{center}
        \begin{boxedverbatim}
-(na)->h-3 - (0.00306937)
\end{boxedverbatim}
      \end{center}
      The level of indentation indicates the rank of this isotope in
      the tree.  This can be best seen by viewing the whole file and
      noting the line's relative indentation.  The information given
      in such an entry is as follows:
      \begin{description}
      \item[reaction type: (na)] This indicates the reaction type(s).
        If multiple reactions lead to this product, the reactions will
        be separated by commas.  The information indicates the emitted
        particles only.  Therefore, in this example, the reaction is
        an (n,na) reaction.  Generally, standard symbols are used,
        such as `n' for neutrons, `a' for alpha particles, `p',`d',`t'
        for the three isotopes of hydrogen, respectively, and `h' for
        helium-3. For all neutron reactions, an additional `*' is used
        to indicate that the product is in an excited isomeric state.
        Finally, for decay reactions the symbol `*D' is used.
      \item[product nuclide: h-3] The product isotope's chemical
        symbol and atomic number.  In cases where the product is in an
        isomeric state, this will be followed by a letter (m,n,...)
        indicating which isomeric state.
      \item[truncation mode: -] This single character indicates the result
        of the truncation calculation at this node.  There are five possible
        results as follows (see Chapter \ref{chap:physical}):
        \begin{description}
        \item[-] This code indicates that the chain continues normally
          because this isotope passed all the tests.
        \item[*] This code indicates that only the radioactive decays of the
          chain will be followed after this node.  This arises when the
          production does not pass the truncation tolerance test, but
          ensures that the result includes all the radioactive products.
          Stable products which are descendants of this node may be
          calculated if they themselves pass the ignore tolerance test.
        \item[/] This code indicates that the chain will be fully
          truncated at this node, and the result will include this
          node.  This arises when the node is a stable isotope and
          does not pass the truncation tolerance test, but does pass
          the ignore tolerance test.
        \item[$<$] This code indicates that the chain will be fully
          truncated at this node and will \textsl{not} be included in
          the result.  This arises when the production of this nuclide
          does not pass either the truncation or the ignore tolerance
          test.
        \end{description}
      \item[truncation production: (0.00306937) ] This indicates the
        relative production at the end of operation of this nuclide
        from the initial isotope during the truncation calculation.
        As explained in Chapter \ref{chap:physical}, this represents
        the total production of this nuclide during the whole problem,
        assuming that none of it is transmuted or decays further.  If
        this production is not calculated, for example, because the
        chain is only being followed on radioactive reactions and this
        nuclide is stable, then this entry will be ` - '.
      \end{description}
      
    \end{subsection}
    
  \end{section}


  \begin{section}{Binary Reaction Library Format}\label{app:binary_data}
    
    Because the reaction schemes/chains are created by a depth first
    search using the data from the transmutation and decay libraries,
    these libraries need to be accessed extensively and randomly.  In
    the past, such random access was not possible due to limits on
    mass storage devices.  Currently, in a text format, such random
    access would still be very tedious.  To ensure that this random
    access does not create a drag on ALARA, it is necessary to either
    store the entire library in memory or use a binary file format.
    Because the libraries are often quite large (many MB) a simple
    binary format was designed.  This section will describe the
    formats for the binary files and their indexes, generated in a
    text format and then appended in binary format to the end of the
    binary library.
    
    Note that all cross-sections have one more group than the number
    of neutron groups in the library.  This last ``group'' is used to
    store the decay rate for this reaction: the product of the total
    decay rate and the branching ratio (zero for many cases).
    
    The binary file format is described in figure
    \ref{fig:binary_format} by listing, in order, the data written to
    the file using the format: \textsl{(data
      type)}Description[\textbf{size}].

    \begin{figure}[htbp]
      \begin{center}
        \epsfig{file=eps/bin_fmt.eps}
      \caption{The format description of an \ALARA\ V. 2 binary library.}
      \label{fig:binary_format}
      \end{center}
    \end{figure}

    The library used for a reverse problem is identical in format,
    but for the addition of one entry.  Immediately following the
    average decay energy data for the parent isotope is an entry for
    the parent isotope's total destruction rate:
    \begin{center}
      \textsl{(float)} Total destruction cross-section \textbf{[G+1]}
    \end{center}
  \end{section}

  
  \begin{section}{Material and Element Library Formats}\label{app:mat_el_libs}
    
    For the user convenience, ALARA uses both a material and element
    library.  The element library simply contains the natural isotopic
    breakdown of all the elements.  The material library contains the
    elemental breakdown (using natural elemental compositions) of
    well-known materials.
    
    \begin{subsection}{Material Library\label{app:user.matlib}}
      
      \begin{center}
        \epsfig{file=eps/matlib_fmt.eps}
      \end{center}

    \end{subsection}

    \begin{subsection}{Element Library\label{app:user.elelib}}
      
      \begin{center}
        \epsfig{file=eps/elelib_fmt.eps}
      \end{center}

    \end{subsection}

  \end{section}


  \begin{section}{Error Messages}\label{app:errors}
    \renewcommand{\baselinestretch}{1.12}\normalsize
    \begin{description}
    \item[-1:]\textbf{Memory allocation error: $<\!\!string\!\!>$}\ \\
      An error in the runtime allocation of memory occured.
      ``$<\!\!string\!\!>$'' reports the function and variable where
      the error occurred.
    \item[0:]\textbf{Option $<\!\!string\!\!>$ is not implemented yet.} \ \\
      An unsupported command-line option was specified: $string$.
    \item[1:]\textbf{Only one input filename can be specified:
        $<\!\!string\!\!>$.}\ \\
      There appears to be more than one input filename on the
      command-line.  This may be caused by an error in the other
      command line options, or a missing option.
    \end{description}
    
    \begin{subsection}{Input Phase}
      
      Note that all error messages which occur during the input phase
      may not report the accurate cause of the error.  If there is an
      error in the input file, \ALARA\ may not immediately recognize
      the error and then report an error during some later input
      block.  This is particularly true during the first step, reading
      the input file itself.
      
      \begin{subsubsection}{Read Input File}
        \begin{description}
        \item[100:]\textbf{Invalid token in input file:
            $<\!\!string\!\!>$}\ \\
          There is an error in the input file causing it to read an
          invalid keyword.
        \item[101:]\textbf{Unable to open included file:
            '$<\!\!string\!\!>$'.}\ \\
          The file $string$ included in one of the input files can not
          be openned.
        \item[110:]\textbf{Unable to open material library:
            $<\!\!string\!\!>$}\ \\
          The file $string$ specified in the \texttt{material\_lib} input
          block cannot be openned.
        \item[111:]\textbf{Unable to open element library: $<\!\!string\!\!>$}\ \\
          The file $string$ specified in the \texttt{element\_lib} input
          block cannot be openned.
        \item[120:]\textbf{Invalid units in cooling time: \%10g \%c}\ \\
          The specified cooling time does not have one of the supported
          time units.
        \item[121:]\textbf{No after-shutdown/cooling times were defined.}\
          \\
          The \texttt{cooling} input block contains no information before
          the \texttt{end} keyword.
        \item[130:]\textbf{Invalid dimension type: $<\!\!string\!\!>$}\ \\
          The type of dimension, $string$, declared in the \texttt{dimension} block
          is not supported.
        \item[131:]\textbf{Dimension has no boundaries}\ \\
          The \texttt{dimension} block has no zone boundary information
          before the \texttt{end} keyword.
        \item[140:]\textbf{Invalid flux type: $<\!\!string\!\!>$}\ \\
          The flux type, $string$, specified in the \texttt{flux} block in
          not supported.
        \item[150:]\textbf{Invalid geometry type: $<\!\!string\!\!>$}\ \\
          The geometry type, $string$, specified in the \texttt{geometry}
          block is not supported.
        \item[160:]\textbf{History $<\!\!string\!\!>$ is empty} \ \\
          The \texttt{history} input block, $string$, contains no
          information before the \texttt{end} keyword.
        \item[170:]\textbf{Material Loading is empty.}\ \\
          The \texttt{mat\_loading} input block contains no information
          before the \texttt{end} keyword.
        \item[180:]\textbf{Target materials for reverse calculations can only be
            elements or isotopes and not '$<\!\!string\!\!>$'}\ \\
          The component type, $string$, given for this target material
          is not supported.  It must be either ``element'' or ``isostope''.
        \item[181:]\textbf{Invalid material constituent:
            $<\!\!string\!\!>$}\ \\
          The component type, $string$, specified for this mixture
          component is not supported.
        \item[182:]\textbf{Mixture $<\!\!string\!\!>$ has no
            components}\ \\
          The \texttt{mixture} input block, $string$, contains no
          information before the \texttt{end} keyword.
        \item[190:]\textbf{Invalid units in pulse level: \%10g \%c}\
          \\
          The specified pulse level decay time does not have one of
          the supported time units.
        \item[200:]\textbf{Schedule $<\!\!string\!\!>$ is empty} \ 
          \\
          The \texttt{schedule} input block, $string$, contains no
          information before the \texttt{end} keyword.
        \item[210:]\textbf{Invalid units in schedule item delay
            time:  \%10g \%c}\ \\
          The specified inter-schedule delay time does not have one
          of the supported time units.
        \item[211:]\textbf{Invalid units in single pulse time: \%10g \%c}\ \\
          The specified pulse length does not have one of the
          supported time units.
        \item[230:]\textbf{Output type '$<\!\!string\!\!>$' is not
            currently supported.}\ \\
          The output type, $string$, specified for this output
          format is not supported.
        \item[240:]\textbf{Unable to open dump file
            $<\!\!string\!\!>$}\ \\
          The output ``dump'' file could not be openned.
        \end{description}
      \end{subsubsection}
      
      \begin{subsubsection}{Input Checking}
        \begin{description}
        \item[300:]\textbf{Cannot define both zone dimensions and
            interval volumes.}\ \\
          \ALARA\ does not permit the geometry to be defined with both
          the \texttt{dimension} input block and the \texttt{volumes}
          input block.  This would result in redundant and possibly
          inconsistent input.
        \item[301:]\textbf{A material loading is given for more
            zones (\%d) than are defined by the zone dimensions
            (\%d).  Those extra
            zones are being ignored.}\ \\
          The number of zones as defined by the
          \texttt{mat\_loading} input block does is larger than the
          number defined by the \texttt{dimension} blocks.  This is
          permissible, but may lead to dubious results.  The extra
          zones from the \texttt{mat\_loading} block will be ignored.
        \item[302:]\textbf{Number of zones defined by zone
            dimensions (\%d) matches number of material
            loadings defined.(\%d)"}\ \\
          The number of zones as defined by the
          \texttt{mat\_loading} input block does is smaller than the
          number defined by the \texttt{dimension} blocks.  This is
          NOT permissible as it would leave some zones unfilled.
        \item[303:]\textbf{Must define either zone dimensions or
            interval volumes for multi-point problems.}\ \\
          \ALARA\ requires a definition of the geomery using either
          the \texttt{dimension} input block or the \texttt{volumes}
          input block for problems in more than 0 dimensions.
        \item[310:]\textbf{Could not find element $<\!\!string\!\!>$
            in element library.}\ \\
          The element $string$ was not found in the element library.
          This could be due to an error in the material library,
          incorrect user input, or an omission in the element
          library.
        \item[311:]\textbf{Could not find material
            $<\!\!string\!\!>$ in material library.}\ \\
          The material $string$ was not found in the material
          library.  This could be due to incorrect user input or an
          omission in the element library.
        \item[330:]\textbf{Duplicate dimensions of type
            $<\!\!string\!\!>$.}\ \\
          The dimension $string$ was defined more that once in the
          input file.
        \item[331:]\textbf{$<\!\!string1\!\!>$ geometries don't have
            dimensions of type $<\!\!string2\!\!>$.}\ \\
          The dimension type $string2$ was defined for geometry type
          $string1$, which does not allow this kind of dimension.
        \item[340:]\textbf{Unable to open flux file
            $<\!\!string1\!\!>$ for flux $<\!\!string2\!\!>$.}\ \\
          In the flux definition $string2$ the given flux file
          $string1$ cannot be openned.
        \item[350:]\textbf{Toroidal problems with zone dimensions
            require a major radius.}\ \\
          All problems defined as having toroidal geometries must
          define a major radius with the \texttt{major\_radius} input
          block.
        \item[351:]\textbf{Toroidal problems with zone dimensions
            require either a minor radius or a radius dimension.}\ \\
          All problems defined as having toroidal geometries must
          define a minor radius with either a \texttt{dimension} block
          or the \texttt{minor\_radius} input block.
        \item[370:]\textbf{Zone $<\!\!string1\!\!>$ is loaded with a
            non-existent mixture: $<\!\!string2\!\!>$}\ \\
          The mixture $string2$ specified to fill zone $string1$ in
          the \texttt{mat\_loading} block is not defined in the input
          file.  Either add a new mixture definition or change the
          name of the mixture to be used for this zone.
        \item[380:]\textbf{Component type 'l' of mixture
            $<\!\!string1\!\!>$ references a non-existent
            mixture: $<\!\!string2\!\!>$}\ \\
          The mixture $string2$ specified in the ``similar'' component
          of mixture $string1$ is not defined in the input file.
          Either add a new mixture definition or change the name of
          the mixture to be used for this definition.
        \item[400:]\textbf{Unable to find top level schedule.  A top
            level schedule must not used as a sub-schedule.}\ \\
          All of the defined schedules are referenced as sub-schedules
          of other schedules.  This means that there is no top to the
          hierarchical schedule system, as required.
        \item[410:]\textbf{Flux $<\!\!string1\!\!>$ for simple pulse
            item of schedule $<\!\!string2\!\!>$ does not exist.}\ \\
          The flux $string1$ required to calculate the simple pulsing
          schedule item of schedule $string2$ is not defined.
        \item[411:]\textbf{Bad flux file for flux $<\!\!string\!\!>$
            for simple pulse item of schedule $<\!\!string\!\!>$.}\ \\
          The file for flux $string1$ required to calculate the simple
          pulsing schedule item of schedule $string2$ cannot be openned.
        \item[412:]\textbf{Schedule recursion: $<\!\!string\!\!>$.}\ 
          \\
          There is a loop in the schedule hierarchy.  This implies an
          infinitely long and infinitely complex total irradiation
          history, which is unphysical.  Check the definition of the
          schedules.
        \item[413:]\textbf{Schedule $<\!\!string1\!\!>$ for subschedule
            item of schedule $<\!\!string2\!\!>$ does not exist.}\ \\
          The sub-schedule $string1$ defined as a schedule item of
          schedule $string2$ has not been defined.
        \item[414:]\textbf{Pulse history $<\!\!string1\!\!>$ for
            item of schedule $<\!\!string2\!\!>$ does not exist.}\ \\
          The pulsing history $string1$ required to calculate a
          schedule item of schedule $string2$ has not been defined.
        \item[420:]\textbf{Zone $<\!\!string\!\!>$ specified in
            interval volumes was not found in the material loading.}\ 
          \\
          The zone $string$ specified to contain one of the volumes in
          the \texttt{volumes} input block does not exist.
        \item[440:]\textbf{ALARA now requires a binary dump file.
            Openning the default file 'alara.dmp'.}\ \\
          \ALARA\ uses a binary file to store intermediate results.
          You can set the name of this file using the
          \texttt{dump\_file} input block.  Otherwise, the default is
          used.
        \item[441:]\textbf{Unable to open dump file 'alara.dmp'.}\ \\
          The default output dump file could not be openned.
        \end{description}
      \end{subsubsection}
      
      \begin{subsubsection}{Input Cross-referencing}
        \begin{description}
        \item[580:]\textbf{Removing mixture $<\!\!string\!\!>$ not
            used in any zones.}\ \\
          Mixture $string$ was defined in the input file, but is not
          used in any zones.  It's definition is being removed.
        \item[620:]\textbf{You have specified too few normalizations.
            If you specifiy any normalizations, you must specify one
            for each interval.}\ \\
          The \texttt{spatial\_norm} input block must contain an entry
          for each of the fine mesh intervals.  It is not permissible
          to have too few.
        \item[621:]\textbf{You have specified too many normalizations.
            Extra normalizations will be ignored.}\ \\
          It permissible to define too many spatial normalizations,
          but the results may by dubious.  The extra normalizations
          will be ignored.
        \item[622:]\textbf{Flux file $<\!\!string\!\!>$ does not
            contain enough data.}\ \\
          The flux file $string$ does not contain enough data to
          provide a flux for each of the fine mesh intervals.
        \end{description}    
      \end{subsubsection}
    \end{subsection}
    
    \begin{subsection}{Data Library Errors}
      
      \begin{description}
      \item[1000:]\textbf{Data library type $<\!\!string\!\!>$ (\%d) is
          not yet supported.}\ \\
        The specified library type $string$ is not supported.
      \item[1001:]\textbf{Conversion from $<\!\!string1\!\!>$ (\%d) to
          $<\!\!string2\!\!>$ (\%d) is not yet supported.}\ \\
        Conversion between the specified library types $string1$ and
        $string2$ is not supported.
      \item[1001:]\textbf{Conversion from $<\!\!string\!\!>$ (\%d) to
          (\%d) is not yet supported.}\ \\
        Conversion between the specified library types $string1$ and
        $\%d$ is not supported.
      \item[1100:]\textbf{You have specified library type 'alaralib' but
          given the filename of an 'adjlib' library.}\ \\
        The type of library specified in the input block must match the
        internally recorded library type.
      \item[1101:]\textbf{You have specified library type 'alaralib' but
          given the filename of an unidentified library.}\ \\
        The type of library specified in the input block must match the
        internally recorded library type.
      \item[1102:]\textbf{You have specified library type 'adjlib' but
          given the filename of an 'alaralib' library.}\ \\
        The type of library specified in the input block must match the
        internally recorded library type.
      \item[1103:]\textbf{You have specified library type 'adjlib' but
          given the filename of an unidentified library.}\ \\
        The type of library specified in the input block must match the
        internally recorded library type.
      \end{description}    
    \end{subsection}
    
    \begin{subsection}{Programming Errors}
      
      In some places, if \ALARA\ reaches that point in the program, it
      implies an error in the logic of the code.  Please report such
      errors to the code author.
      
      \begin{description}
      \item[9000:]\textbf{Programming Error:...}
      \end{description}    
      
    \end{subsection}

  \end{section}
  
    
\end{chapter}
  
 


\begin{chapter}{Selected  $C\!\!\!\stackrel{+\!\!+}{}$ header files}
  \label{app:headers}

\end{chapter}

