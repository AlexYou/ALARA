
\begin{chapter}{Users' Guide\label{app:user.guide}}
  
  The usage of ALARA is fairly straightforward, requiring little
  knowledge of the inner workings of the code.  Of course, to ensure
  that ALARA is well-suited to the problems that you are trying to
  solve, you are encouraged to read the chapters \ref{chap:physical}
  and \ref{chap:math} and understand the physical and mathematical
  modeling characteristics of ALARA.
  
  This chapter will describe the command-line options of ALARA and
  then describe the basic support files which are necessary to run
  ALARA.
  
  \begin{section}{Command-line Options}\label{app:user.cmd}
    
    \begin{center}
      \textbf{\texttt{alara [-t }\textsl{tree\_output\_filename}\texttt{]
          [-h] [-v }\textsl{[n]}\texttt{]} \textsl{input\_filename}}
    \end{center}
    ALARA currently supports 4 command-line options:
    \begin{description}
    \item[\texttt{-h} Help]\ \\
      This option will print a short help message describing the
      command-line options and their usage.
    \item[\texttt{-t }\textsl{tree\_output\_filename} Tree File]\ \\
      This option allows you to define the name of the output file for the
      tree creation and truncation information.  This information can
      later be used for basic pathway analysis.  The default is to create
      no tree file.  The format of this file is described in section
      \ref{app:user.output.tree}.
    \item[\texttt{-v }\textsl{[n]} Verbose]\ \\
      This option alters the verbosity of the output.  Without this
      option, only the final results will be displayed.  By using this
      option, some of the details of the calculation are included in the
      output.  The level of detail is controlled by the optional value,
      \textsl{n}, which has a value between 1 (least detail) and 5 (most
      detail).  If no value is given, it defaults to 1.
    \item[\textsl{input\_filename} Input File]\ \\
      This option allows you to define the name of the input file to be
      used by ALARA.  If no name is specified, the input will be read from
      stdin.
    \end{description}
    
  \end{section}
  
  
  \begin{section}{Input File Description\label{app:user.input}}
    
    The input file for ALARA has been designed to ensure that the
    input information is easy to understand, edit and comment.  This
    is possible by using a very free format which permits comments,
    blank lines, inclusion of other files, and arbitrary ordering of
    the input information.  After reading the full input file, \ALARA\
    performs various cross-checks and cross-references to ensure that
    the input data is self-consistent.  It then goes on to preprocess
    the data for the calculation.  All attempts have been made to give
    useful error messages when the data is not consistent.

    \begin{subsection}{General}
    
      There are 19 possible input block types, which can appear in any
      order and many blocks can occur more than once, if needed.  One
      block type, \texttt{convert\_lib}, is only used to convert data
      library formats, and will cause \ALARA\ to halt if used in an
      input file.  The other 18 input blocks are:

      {
        \renewcommand{\baselinestretch}{1}\normalsize
        \noindent\begin{tabular}{p{2in}p{2in}p{2in}}
          \begin{enumerate}
          \item \texttt{geometry}
          \item \texttt{dimension}
          \item \texttt{major\_radius}
          \item \texttt{minor\_radius}
          \item \texttt{volumes}
          \item \texttt{mat\_loading}
          \end{enumerate} &
          \begin{enumerate}\setcounter{enumi}{6}
          \item \texttt{mixture}
          \item \texttt{flux}
          \item \texttt{spatial\_norm}
          \item \texttt{schedule}
          \item \texttt{pulsehistory}
          \item \texttt{truncation}
          \end{enumerate} &
          \begin{enumerate}\setcounter{enumi}{12}
          \item \texttt{output}
          \item \texttt{cooling}
          \item \texttt{material\_lib}
          \item \texttt{element\_lib}
          \item \texttt{data\_library}
          \item \texttt{dump\_file}
          \end{enumerate}
        \end{tabular}
        }

      Not all input blocks are required, with some having default
      values and others being unnecessary for certain problems.  There
      are also some input blocks which are incompatible with each
      other.  While superfluous input blocks may go unnoticed (there
      are occasional warnings), incompatible input blocks will create
      an error.

      Most of the input blocks allow the user to define their own
      symbolic names for cross-referencing the various input data.
      Any string of characters can be used as long as its does not
      contain any whitespace (spaces, tabs, new-lines, etc.).  It is
      considered dangerous, however, to use one of the keywords as a
      symbolic name.  If the input file is correct, it will function
      properly, but if there are errors in the input file, the usage
      of keywords as symbolic names may make the error message
      irrelevent.  The keywords include those listed in the above list
      and the keyword ``end''.  While many input blocks of fixed
      length require nothing to indicate the end of the block, some
      blocks have a variable length and require the keyword ``end'' to
      terminate the block.
      
      Some input elements are times which can be defined in a number
      of different units.  When this is the case, the floating point
      time value should be followed by a single character representing
      one of the following units:

      \begin{center}
        \renewcommand{\baselinestretch}{1}\normalsize
        \begin{tabular}{|l|l|l|}
          \hline
          \texttt{[s]econd} & 
          \texttt{[m]inute} = 60 seconds& 
          \texttt{[h]our} = 60 minutes\\\hline
          \texttt{[d]ay} = 24 hours& 
          \texttt{[w]eek} = 7 days& 
          \texttt{[y]ear} = 52 weeks\\\hline
          \multicolumn{3}{|c|}{\texttt{[c]entury} = 100 years}\\\hline
        \end{tabular}
      \end{center}

      One input file can be included in another with the
      \texttt{\#include} directive, similar to the C programming
      language.  Any number of files can be included, and included
      files can also contain directives to include other files.  The
      only restriction is that the inclusion must not occur within an
      input block!
      
      All other lines in which the first non-space character is the
      pound sign (or number sign) (\#) are considered as comments.
      Comments can also be used after any single word input (an input
      value which has no whitespace) by using the same comment
      character (\#).  Such comments extend to the end of the current
      line.  Blank lines are permitted anywhere in the input file.
      
      When length units are implied in the input for sizes and
      dimensions, it is only important that all implied units be
      consistent but not what unit is implied.
    \end{subsection}
    
    \begin{subsection}{\texttt{geometry}\label{app:user.input.geom}}
    
      This input block only necessary when defining a geometry using
      the \texttt{dimension} input block, but may always be included.
      It should only occur once.  This input block takes a single
      argument which must be one of the following:
      \begin{center}
        \texttt{point | rectangular | cylindrical | spherical | torus}
      \end{center}
      This input block should not be terminated.

      \begin{center}
        \renewcommand{\baselinestretch}{1}\normalsize
        \begin{boxedverbatim}
# problem geometry
geometry spherical          
\end{boxedverbatim}
      \end{center}

      If using the \texttt{dimension} input block to define the
      geometry and the type is \texttt{torus}, the
      \texttt{major\_radius} input block is required and the
      \texttt{minor\_radius} block may also be required.
    \end{subsection}

    \begin{subsection}{\texttt{dimension}\label{app:user.input.dim}}
      
      This input block is used to define the layout of the geometry,
      and should be included once for each dimension needed in the
      problem.  The first element of the dimension block is an
      indicator of which dimension is being defined and should be one
      of the following:
      \begin{center}
        \texttt{ x | y | z | r | theta | phi }
      \end{center}
      \ALARA\ will check to ensure that only dimensions relevant to
      the defined geometry are included.  For example, defining the
      '\texttt{x}' dimension in a spherical problem will generate an
      error.

      The next element of the dimension block is the lower boundary of
      the first zone, expressed as a floating point number.  This is
      followed by a list of pairs, one pair for each zone: an integer
      specifying the number of intervals in this zone in this
      dimension and a floating point number indicating the upper
      boundary of the zone.  This list is terminated with the
      \texttt{end} keyword.
      
      \begin{center}
        \renewcommand{\baselinestretch}{1}\normalsize
        \begin{boxedverbatim}
# sample dimension for spherical problem 
dimension r 1.0      # inside radius 
         10 2.0      # 10 intervals in the first zone (1.0,2.0)
         15 3.0      # 15 intervals in the next zone (2.0,3.0) 
end
\end{boxedverbatim}
      \end{center}

      Since this method of defining the geometry calculates the zone
      membership and volume of the fine mesh intervals from the
      \texttt{dimension} data, it is incompatible with the
      \texttt{volumes} input block.  Including both will generate an
      error message.
    \end{subsection}

    \begin{subsection}{\texttt{major\_radius} and  \texttt{minor\_radius} \label{app:user.input.tor_radii}}
      
      These two input blocks are used to define the major and minor
      radii of toroidal geometries.  They are only needed if defining
      a toroidal geometry with \texttt{dimension} input blocks, and
      each should only be included once.  Furthermore, if the minor
      radius dimension is defined with a \texttt{dimension} block, the
      \texttt{minor\_radius} input block is not required.  In both
      cases, these input blocks have a fixed size, with a single
      argument specifying the radius as a floating point number.

      \begin{center}
        \renewcommand{\baselinestretch}{1}\normalsize
        \begin{boxedverbatim}
# major and minor radii of torus
major_radius     2.0
minor_radius     0.5          
\end{boxedverbatim}
      \end{center}

    \end{subsection}

    \begin{subsection}{\texttt{volumes}\label{app:user.input.vol}}
      
      This input block is used to define the volumes and zone
      membership of the fine mesh intervals of the problem.  This
      block can be used instead of the \texttt{dimension} method of
      defining the geometry.  If both are used, an error will result.
      This block should only occur once.  Multiple occurences will
      result in undefined behaviour.

      This input block should be a list of pairs, one pair for each
      interval.  Each pair consists of a floating point value for the
      volume of that interval and the symbolic name of the zone which
      contains that interval.  These symbolic names should correspond
      with the symbolic names given to the zones in the
      \texttt{mat\_loading} input block (see section
      \ref{app:user.input.loading}).  This list must be terminated with the
      keyword \texttt{end}.

      \begin{center}
        \renewcommand{\baselinestretch}{1}\normalsize
        \begin{boxedverbatim}
# list of fine mesh intervals
volumes
   0.5     vacuum_vessel
   1.5     shield_zone
   2.34    blanket_zone
   1.92    first_wall
end          
\end{boxedverbatim}
      \end{center}
    \end{subsection}

    \begin{subsection}{\texttt{mat\_loading}\label{app:user.input.loading}}
      
      This input block is used to indicate which mixtures are
      contained in each zone.  This block is a list with one pair of
      entries for every zone.  Each pair consists of a symbolic name
      for the zone and a symbolic name for the mixture contained in
      that zone.  This list is terminated by the keyword \texttt{end}.
      This block should only occur once.  Multiple occurences will
      result in undefined behaviour.

      If the geometry is defined using the \texttt{dimension} input
      blocks, there number of zones defined here must match the number
      of zones defined in the \texttt{dimension} blocks exactly; if
      not, an error results.  If the \texttt{volumes} method is used
      to define the geometry, this block uniquely determines the
      number of zones.  The symbolic name for the mixture must match
      one of the \texttt{mixture} definitions exactly, or be the
      keyword '\texttt{void}', indicating that this zone is empty of
      material.

      \begin{center}
        \renewcommand{\baselinestretch}{1}\normalsize
        \begin{boxedverbatim}
# material loadings for all zones
mat_loading
   vaccum_vessel  VV_materials
   shield_zone    shield_mixture
   blanket_zone   breeding_blanket
   first_wall     liquid_FW
   scrapeoff      void          
end
\end{boxedverbatim}
      \end{center}
      
    \end{subsection}
 

    \begin{subsection}{\texttt{mixture}\label{app:user.input.mix}}
      
      This kind of block is used to define the composition of a
      mixture.  This block can occur as many times as necessary to
      define all the mixture compositions in the problem.  Any
      mixtures that are defined, but not used in the problem will
      generate a warning and be removed from the list of mixtures.
      
      The first element of a \texttt{mixture} block is the symbolic
      name used to refer to this mixture elsewhere in the input file.
      Following this is a list of triplets with one triplet for each
      component of the mixture.  The list must be terminated with the
      keyword '\texttt{end}'.  The first element of each triplet
      describes what the type of that component and should be one of:
      \begin{center}
        \texttt{material | element | isotope | like | target}
      \end{center}

      The interpretation of the remaining elements in each triplet are
      based on this first element:
      \begin{description}
      \item[\texttt{material}] The second element in this triplet is
        the symbolic name of a material definition which exists in the
        material library (see section \ref{app:user.matlib}).  The
        final element is a floating point value representing the
        relative density of this material.  This value, usually
        between 0 and 1, will be multiplied by the density found in
        the material library to define the density of this component.
        It can also be interpreted as the volume fraction of this
        material in this mixture.
      \item[\texttt{element}] The second element in this triplet is
        the chemical symbol of the element.  This element will be
        expanded into a list of isotopes using the natural isotopic
        abundances found in the element library (see section
        \ref{app:user.elelib}).  The final element is a floating
        point value representing the relative density of this
        material.  This value, usually between 0 and 1, will be
        multiplied by the standard theoretical density found in the
        element library to define the density of this component.  It
        can also be interpreted as the volume fraction of this
        element in this mixture.
      \item[\texttt{isotope}] The second element in this triplet is a
        symbolic name for the isotope in the format ZZ-AAA, where ZZ
        is the chemical symbol and AAA is the mass number, for
        example, \texttt{i-127, ca-40} or \texttt{pb-207}.  The final
        element of this triplet is the number density of this isotope
        in the mixture.
      \item[\texttt{like}] This type of entry is provided as a
        convenience and indicates that this component is like another
        user-defined mixture, with a potentially different density.
        The second element of this triplet is the symbolic name of
        another mixture definition.  If the other mixture definition
        is not found, an error will result.  The final element of the
        triplet is a relative density, which will be used to normalize
        the density as defined in that mixture's own definition.
        This might be used when a user-defined mixture makes up part
        of another mixture. [Hint: it is permissable to define a
        mixture that is not used in any zones, but only used as part
        of another mixture.]
      \item[\texttt{target}] This type of entry is used to initiate a
        reverse calculation (see section \ref{sec:physical.reverse})
        and define the target isotopes for the reverse calculation.
        The user can define an arbitrary number of target isotopes.
        The second element of this triplet is one of the keywords
        \texttt{element} or \texttt{isotope}, indicating what kind of
        target this is.  The final element is the symbolic name of
        either the \texttt{element} or \texttt{isotope}, as defined in
        the corresponding entries in this list above.  There is no
        density in this type of entry.  If a target is of type
        \texttt{element}, the element will be expanded using the
        element library to create a list of isotopes.
      \end{description}
 
      \begin{center}
        \renewcommand{\baselinestretch}{1}\normalsize
        \begin{boxedverbatim}
# definition of vacuum vessel
mixture VV_materials
   material SS316-L 1.0
end

# definition of fusion shield
mixture shield_mixture
   like     vacuum_vessel  0.6
   element  he             0.4
end

# definition of fusion breeding blanket
mixture breeding_blanket
   element  li      0.95
   isotope  li-6    6.02e23
   target   isotope h-3
end
\end{boxedverbatim}
      \end{center}

      Note that even if a target is defined in only one mixture, it
      will cause the whole problem to be run as a reverse problem.
      There is therefore little purpose in having mixture definitions
      without targets (such as in this example).
    \end{subsection}

    \begin{subsection}{\texttt{flux}\label{app:user.input.flux}}
      
      This input block defines a set of flux spectra.  Since different
      parts of the irradiation history can have different flux
      spectra, this block may occur as many times as necessary to
      represent all the different necessary flux definitions.  The
      first element of this block is a symbolic name which will be
      used to refer to this flux spectra definition.  The other
      elements of this block are a filename, a floating point scalar
      normalization, an integer skip value (see below), and flux type
      indicator string, respectively.

      The flux filename should indicate which file contains this flux
      information, including path information appropriate to find the
      file from the directory in which \ALARA\ will be run.

      The scalar normalization permits the scaling of the flux at all
      spatial points (as opposed to the \texttt{spatial\_norm}
      information in the next section).  All groups of all fluxes in
      this definition will be multiplied by this value.
      
      The skip value indicates how many N-group flux entries to skip
      in this file before reading the first flux.  This permits the
      user to have one file with many different flux spectra.  For
      example, if the schedule requires two different flux spectra for
      V different fine mesh points, the data for the first one may be
      at the beginning of the file, with a skip of 0, while the data
      for the second flux definition would be after these first
      fluxes, with a skip of V.
      
      The last element is a character string indicating the format of
      the flux file.  Currently the only supported format is
      \texttt{default}.

      \begin{center}
        \renewcommand{\baselinestretch}{1}\normalsize
        \begin{boxedverbatim}
# flux for first part of irradiation schedule
flux high_yield_flux ../n.transport/machine.flux 1.0 0 default

# flux for second part of irradiation schedule
flux low_yeild_flux ../n.transport/machine.flux 1.0 432 default

# flux for final part of irradiation schedule
flux attenuated_high_yield ../n.transprot/machine.flux 0.5 0 default
\end{boxedverbatim}
      \end{center}
      
      [Hint: Different flux definitions might use exactly the same
      flux values (same flux file and skip value) but a different
      scaling value.]
    \end{subsection}

    \begin{subsection}{\texttt{spatial\_norm}\label{app:user.input.norm}}
      
      This input block allows the user to specify a scalar flux
      normalization for each fine mesh interval, such as might be
      required to renormalize the results of a transport calculation
      on an approximated geometry.  The number of normalizations must
      be at least as many as the number of defined intervals,
      regardless of how the intervals are defined (\texttt{dimension}
      vs. \texttt{volumes}).  If there are too few, an error will
      result; if there are too many, a warning will result.
      
      This block consists of a list of floating point normalization
      values, one value for each interval, and requires the
      \texttt{end} keyword to terminate the list.

      \begin{center}
        \renewcommand{\baselinestretch}{1}\normalsize
        \begin{boxedverbatim}
# normalizations to convert cylindrical model
# to toroidal equivalent
spatial_norm
   0.8
   0.85
   0.9
   1.0
   1.03
   1.08
   1.1
end
\end{boxedverbatim}
      \end{center}

      [Hint: if these values are purely a function of problem
      geometry, and not mixture composition, it is possible that many
      problems have the same spatial normalization.  Put this data in
      a separate file and \texttt{\#include} it when you need it.]
      
    \end{subsection}
   
    \begin{subsection}{\texttt{schedule}\label{app:user.input.sched}}
      
      This kind of block is used to define a single schedule in the
      hierarchy of the full irradiation history.  Since the hierarchy
      may be composed of many schedules, this block might occur many
      times.  The first element in this input block is a symbolic name
      by which this schedule can be refered to.  Following this is a
      list of items which occur in this schedule.  There are two
      possible types for each item, and their may be an arbitrary list
      of items in a schedule.  This list must be terminated with the
      keyword '\texttt{end}'.  See section \ref{app:user.schedules}
      for more information about defining schedules.
      
      The first type of item is a simple pulse and the entries for
      this kind of item are a floating point operating time, a single
      character defining the units of that operating time, a symbolic
      flux name, a symbolic pulsing definition name, a floating point
      post-item delay time, and a single character defining the units
      of that delay time.
      
      The second type of item is a sub-schedule and the entries for
      this kind of item are a symbolic name for the sub-schedule, a
      symbolic pulsing definition name, a floating point post-item delay
      time, and a single character defining the units of that delay
      time.
      
      In both cases, if the symbolicly named items (flux, pulsing
      definition, or schedule) are not found during cross-referencing,
      an error results.

      \begin{center}
        \renewcommand{\baselinestretch}{1}\normalsize
        \begin{boxedverbatim}
# top level schedule
schedule top
    # a schedule of operation defined by 'phase_1_sched' which
    #     will be repeated with a cycle defiend by 'phase_1_cycle'
    #     after which there is a 10 week delay
  phase_1_sched phase_1_cycle 10 w
    # a schedule of operation defined by 'phase_2_sched' which
    #     will be repeated with a cycle defiend by 'phase_2_cycle'
    #     after which there is a 5 week delay
  phase_2_sched phase_2_cycle 5 w
end

# phase 1 schedule
schedule phase_1_sched
    # a pulsing regimen with 1800 second pulses at flux 'high_yield_flux' 
    # pulsed with '5_week_plan_short' after which there is a 1 week delay
  1800 s high_yield_flux 5_week_plan_short 1 w
    # a special schedule for cleaning the facility, 'cleaning_sched'
    # with a pulsing history 'cleaning_cycle' after which there is no delay
  cleaning_sched cleaning_cycle 0 s
end

# cleaning sched
schedule cleaning_sched
    # two consecutive pulsing regimens, each with one day pulses and
    # pulsing history 'daily_week_long' followed by a 1 day delay
    # the first uses 'low_yield_flux' and the other 'low_energy_flux'
  1 d low_yield_flux daily_week_long 1 d
  1 d low_energy_flux daily_week_long 1 d
end

# phase 2 schedule
schedule phase_2_sched
    # a pulsing regimen with 1 hour pulses at flux 
    # 'high_yield_flux' pulsed with '5_week_plan_long' after which
    # there is a 1 week delay
  1 h high_yield_flux 5_week_plan_long 1 w
    # same cleaning phase as in phase 1 schedule
  cleaning_sched cleaning_cycle 0 s
end
\end{boxedverbatim}
      \end{center}

    \end{subsection}

    \begin{subsection}{\texttt{pulsehistory}\label{app:user.input.pulse}}
      
      This kind of input block defines the multi-level pulsing
      histories which are referenced in the \texttt{schedule}
      definitions.  See section \ref{app:user.schedules} for more
      information about defining schedules and histories.  Since many
      different pulsing histories may be used throughout the hierarchy
      of schedules, this block may occur many times.

      The first element of each block is a symbolic name for referring
      to this pulsing schedule.  Following this is a list of pulsing
      level definition triplets, each consisting of an integer number
      of pulses, a floating point delay time between pulses, and a
      single character defining the units of that delay time.  Since
      an arbitrary number of pulsing levels is allowed, this list must
      be terminated with the keyword '\texttt{end}'.

      \begin{center}
        \renewcommand{\baselinestretch}{1}\normalsize
        \begin{boxedverbatim}
# define 5 weeks of the short pulses (1800 s = 1/2 hour)
pulsehistory 5_week_plan_short
   4  90 m   # 4 pulses each day with one every 2 hours
   5  17.5 h # 5 days with 16 hours + 90 minutes delay (overnight)
   5  2.73 d # 5 weeks with 2 days + 17.5 hours delay (weekends)
end

# define 5 weeks of the long pulses (1 h)
pulsehistory 5_week_plan_long
   4  1 h    # 4 pulses each day with one every 2 hours
   5  17 h   # 5 days with 16 hours + 1 h delay (overnight)
   5  2.71 d # 5 weeks with 2 days + 17 hours delay (weekends)
end
\end{boxedverbatim}
      \end{center}

    \end{subsection}

    \begin{subsection}{\texttt{truncation}\label{app:user.input.trunc}}
      
      This fixed sized input block defines the parameters used in
      truncating the activation trees.  See section
      \ref{sec:physical.chains.trunc} for a detailed discussion of the
      tree truncation issue.  The first element of this block is the
      truncation tolerance and the second is an ignore tolerance.
      When testing the relative atom loss (or relative production in
      reverse calculations), any value higher than the truncation
      tolerance will result in continuing the tree while lower values
      will result in truncation.  If the value is also lower than the
      ignore tolerance, that node is completely ignored.

      \begin{center}
        \renewcommand{\baselinestretch}{1}\normalsize
        \begin{boxedverbatim}
# defined a 1/10000 tolerance, ignoring 1e-10
truncation   1e-4    1e-10
\end{boxedverbatim}
      \end{center}
      
    \end{subsection}

    \begin{subsection}{\texttt{output}\label{app:user.input.output}}
      
      This kind of input block allows the user to define the nature
      and format of the output.  The first element of an output format
      block indicates the resolution of the output and should be one
      of:
      \begin{center}
        \texttt{ interval | zone | mixture }
      \end{center}
      This is followed by a list of output types and modifiers which
      are described in the following table:

      \begin{center}
        \renewcommand{\baselinestretch}{1}\normalsize
        \begin{tabular}{|c|l|}
          \hline
          keyword & function \\\hline\hline
          component & component breakdown in addition to total response\\\hline
          number\_density & number density result of all produced isotopes\\\hline
          specific\_activity & specific activity of all radioactive isotopes\\\hline
          total\_heat & total decay heat\\\hline
          alpha\_heat & total alpha heating\\\hline
          beta\_heat & total beta heating\\\hline
          gamma\_heat & total gamma heating\\\hline
        \end{tabular}

        \vspace{2\baselineskip}

        \begin{boxedverbatim}
# output only total zone number densities (no component breakdown)
output zone
   number_density
end

# now output activities and decay heats for zones
#   with component breakdown
output zone
   component
   specific_activity
   total_heat
end

# now output total activities for mixtures
output mixture
   specific_activity
end
\end{boxedverbatim}
      \end{center}      



      
      See section \ref{app:user.output} for information on
      interpreting the output files generated by \ALARA.
    \end{subsection}

    \begin{subsection}{\texttt{cooling}\label{app:user.input.cool}}
      
      This input block is used to define the after-shutdown cooling
      times at which the problem will be solved.  Multiple occurences
      will result in undefined behaviour.  This block is simply a list
      of times, where each time consists of a floating point time
      followed by a single character defining the time's units.  Since
      an arbitrary number of cooling times can be solved, this list
      must be terminated with the keyword '\texttt{end}'.

      \begin{center}
        \renewcommand{\baselinestretch}{1}\normalsize
        \begin{boxedverbatim}
# a wide array of cooling times
cooling
     1 m
     1 d    
     1 w
   0.5 y
     1 y
    10 y
     1 c
end          
\end{boxedverbatim}
      \end{center}

    \end{subsection}

    \begin{subsection}{\texttt{material\_lib} and \texttt{element\_lib}\label{app:user.input.matlibs}}
      
      These two input blocks are used to specify the libraries to be
      used for looking up the definitions of materials and elements
      when they are given as mixture components (see section
      \ref{app:user.input.mix}).  Each block has a single element
      consisting of the filename to be used in each case, including
      appropriate path information to find that file from the
      directory where \ALARA\ is being run.  For more information on
      the format of these libraries, see section
      \ref{app:mat_el_libs}.

      \begin{center}
        \renewcommand{\baselinestretch}{1}\normalsize
        \begin{boxedverbatim}
material_lib  /alara/data/matlib/magnetic_fusion
element_lib   /alara/data/std.elelib          
\end{boxedverbatim}
      \end{center}
    \end{subsection}

    \begin{subsection}{\texttt{data\_library}\label{sec:user.input.datalib}}
      
      This input block is used to define the type and location of the
      nuclear data library.  The first element of this block is
      character string which defines the type of library.  The
      subsequent elements indicate the location of the file.
      Currently accepted library types are:
      \begin{description}
      \item[alaralib] Standard \ALARA\ v.2 binary library.  This
        library type requires a single filename indicating the
        location of the library.
      \item[adjlib] Standard \ALARA\ v.2 reverse library  This
        library type requires a single filename indicating the
        location of the library.
      \item[eaflib] Data library following EAF conventions (ENDF/B).
        This library type requires two filenames, the transmutation
        library and the decay library, respectively.  These libraries
        will be read and processed, creating an \ALARA\ v.2 binary
        library with the name '\texttt{alarabin}' for use in
        subsequent calculations.  Alternatively, this library could be
        converted to an \ALARA\ v.2 binary library as a separate
        process using the \texttt{convert\_lib} function.
      \end{description}

      For both types of \ALARA\ v.2 library, the extension ``.lib''
      will be added to the filename indicated in this input block.
      Otherwise, all filenames should include appropriate path
      information to find the file from the directory in which \ALARA\
      will be run.

      \begin{center}
        \renewcommand{\baselinestretch}{1}\normalsize
        \begin{boxedverbatim}
# convert and use an EAF formated FENDL2 library
data_library eaflib /alara/data.src/FENDL2/fendlg-2.0
       /alara/data.src/FENDL2/fendld-2.0          
\end{boxedverbatim}
      \end{center}

    \end{subsection}

    \begin{subsection}{\texttt{dump\_file}\label{sec:user.input.dump}}
      
      This input block defines the filename to use for the binary data
      dump produced during a run of \ALARA.  This is currently used to
      store the intermediate results during the calculation, and will
      be extended in the future to allow sophisticated post-processing
      of the data.  This filename should be a valid name for a new
      file, including path information appropriate for the directory
      where \ALARA\ will be run.  Note that if the dump file already
      exists, it will be overwritten with no warning.  If this input
      block is omitted, the default name '\texttt{alara.dump}' will be
      used.

      \begin{center}
        \renewcommand{\baselinestretch}{1}\normalsize
        \begin{boxedverbatim}
# define a dump file name
dump_file testing/test_problem.dump          
\end{boxedverbatim}
      \end{center}

    \end{subsection}

  \end{section}
  
  \begin{section}{Defining Irradiation Schedules\label{app:user.schedules}}
    
    With the added flexibility in irradiation schedule definition
    comes added complication.  All attempts have been made for this to
    be a straightforward process, and this section should help make it
    clearer.

  \end{section}

  \begin{section}{\ALARA\ Output File Formats\label{app:user.output}}

    \begin{subsection}{Output File}
      
    \end{subsection}

    \begin{subsection}{Tree File\label{app:user.output.tree}}
      
    \end{subsection}
    
  \end{section}


  \begin{section}{Binary Reaction Library Format}\label{app:binary_data}
  \end{section}
  \begin{section}{Material and Element Library Formats}\label{app:mat_el_libs}

    For the convenience of the user, ALARA uses both a material and
    element library.  The element library simply contains the natural
    isotopic breakdown of all the elements.  The material library
    contains the elemental breakdown (using natural elemental
    compositions) of well-known materials.     
    
    \begin{subsection}{Material Library\label{app:user.matlib}}
      
      \begin{binaryFile}
      \item Material info once for each material
        \begin{binaryFile}
        \item {\em (char)}Material name [no white space allowed in name]
        \item {\em (float)}Material density
        \item {\em (int)}Number of elements in material
        \item Element information once for each element
          \begin{binaryFile}
          \item {\em (char)} Elemental symbol
          \item {\em (float)} Weight fraction of this element
          \item {\em (int)} Atomic number of element
          \end{binaryFile}
        \end{binaryFile}
      \end{binaryFile}
    \end{subsection}

    \begin{subsection}{Element Library\label{app:user.elelib}}
      
      \begin{binaryFile}
      \item Title indicating which element library this is
      \item Element info once for each element
        \begin{binaryFile}
        \item elemental symbol
        \item nominal elemental mass [g/mol]
        \item atomic number
        \item nominal elemental density [g/cm$^3$]
        \item number of constituent isotopes
        \item isotope information once for each naturally occurring isotope
          \begin{binaryFile}
          \item mass number of isotope
          \item atomic abundance of isotope  
          \end{binaryFile}
        \end{binaryFile}
      \end{binaryFile}

    \end{subsection}

  \end{section}


  \begin{section}{Flux File Formats}\label{app:flux_files}
  \end{section}
  \begin{section}{Error Messages}\label{app:errors}
  \end{section}
  

\end{chapter}

